
% Default to the notebook output style

    


% Inherit from the specified cell style.




    
\documentclass[11pt]{article}

    
    
    \usepackage[T1]{fontenc}
    % Nicer default font (+ math font) than Computer Modern for most use cases
    \usepackage{mathpazo}

    % Basic figure setup, for now with no caption control since it's done
    % automatically by Pandoc (which extracts ![](path) syntax from Markdown).
    \usepackage{graphicx}
    % We will generate all images so they have a width \maxwidth. This means
    % that they will get their normal width if they fit onto the page, but
    % are scaled down if they would overflow the margins.
    \makeatletter
    \def\maxwidth{\ifdim\Gin@nat@width>\linewidth\linewidth
    \else\Gin@nat@width\fi}
    \makeatother
    \let\Oldincludegraphics\includegraphics
    % Set max figure width to be 80% of text width, for now hardcoded.
    \renewcommand{\includegraphics}[1]{\Oldincludegraphics[width=.8\maxwidth]{#1}}
    % Ensure that by default, figures have no caption (until we provide a
    % proper Figure object with a Caption API and a way to capture that
    % in the conversion process - todo).
    \usepackage{caption}
    \DeclareCaptionLabelFormat{nolabel}{}
    \captionsetup{labelformat=nolabel}

    \usepackage{adjustbox} % Used to constrain images to a maximum size 
    \usepackage{xcolor} % Allow colors to be defined
    \usepackage{enumerate} % Needed for markdown enumerations to work
    \usepackage{geometry} % Used to adjust the document margins
    \usepackage{amsmath} % Equations
    \usepackage{amssymb} % Equations
    \usepackage{textcomp} % defines textquotesingle
    % Hack from http://tex.stackexchange.com/a/47451/13684:
    \AtBeginDocument{%
        \def\PYZsq{\textquotesingle}% Upright quotes in Pygmentized code
    }
    \usepackage{upquote} % Upright quotes for verbatim code
    \usepackage{eurosym} % defines \euro
    \usepackage[mathletters]{ucs} % Extended unicode (utf-8) support
    \usepackage[utf8x]{inputenc} % Allow utf-8 characters in the tex document
    \usepackage{fancyvrb} % verbatim replacement that allows latex
    \usepackage{grffile} % extends the file name processing of package graphics 
                         % to support a larger range 
    % The hyperref package gives us a pdf with properly built
    % internal navigation ('pdf bookmarks' for the table of contents,
    % internal cross-reference links, web links for URLs, etc.)
    \usepackage{hyperref}
    \usepackage{longtable} % longtable support required by pandoc >1.10
    \usepackage{booktabs}  % table support for pandoc > 1.12.2
    \usepackage[inline]{enumitem} % IRkernel/repr support (it uses the enumerate* environment)
    \usepackage[normalem]{ulem} % ulem is needed to support strikethroughs (\sout)
                                % normalem makes italics be italics, not underlines
    

    
    
    % Colors for the hyperref package
    \definecolor{urlcolor}{rgb}{0,.145,.698}
    \definecolor{linkcolor}{rgb}{.71,0.21,0.01}
    \definecolor{citecolor}{rgb}{.12,.54,.11}

    % ANSI colors
    \definecolor{ansi-black}{HTML}{3E424D}
    \definecolor{ansi-black-intense}{HTML}{282C36}
    \definecolor{ansi-red}{HTML}{E75C58}
    \definecolor{ansi-red-intense}{HTML}{B22B31}
    \definecolor{ansi-green}{HTML}{00A250}
    \definecolor{ansi-green-intense}{HTML}{007427}
    \definecolor{ansi-yellow}{HTML}{DDB62B}
    \definecolor{ansi-yellow-intense}{HTML}{B27D12}
    \definecolor{ansi-blue}{HTML}{208FFB}
    \definecolor{ansi-blue-intense}{HTML}{0065CA}
    \definecolor{ansi-magenta}{HTML}{D160C4}
    \definecolor{ansi-magenta-intense}{HTML}{A03196}
    \definecolor{ansi-cyan}{HTML}{60C6C8}
    \definecolor{ansi-cyan-intense}{HTML}{258F8F}
    \definecolor{ansi-white}{HTML}{C5C1B4}
    \definecolor{ansi-white-intense}{HTML}{A1A6B2}

    % commands and environments needed by pandoc snippets
    % extracted from the output of `pandoc -s`
    \providecommand{\tightlist}{%
      \setlength{\itemsep}{0pt}\setlength{\parskip}{0pt}}
    \DefineVerbatimEnvironment{Highlighting}{Verbatim}{commandchars=\\\{\}}
    % Add ',fontsize=\small' for more characters per line
    \newenvironment{Shaded}{}{}
    \newcommand{\KeywordTok}[1]{\textcolor[rgb]{0.00,0.44,0.13}{\textbf{{#1}}}}
    \newcommand{\DataTypeTok}[1]{\textcolor[rgb]{0.56,0.13,0.00}{{#1}}}
    \newcommand{\DecValTok}[1]{\textcolor[rgb]{0.25,0.63,0.44}{{#1}}}
    \newcommand{\BaseNTok}[1]{\textcolor[rgb]{0.25,0.63,0.44}{{#1}}}
    \newcommand{\FloatTok}[1]{\textcolor[rgb]{0.25,0.63,0.44}{{#1}}}
    \newcommand{\CharTok}[1]{\textcolor[rgb]{0.25,0.44,0.63}{{#1}}}
    \newcommand{\StringTok}[1]{\textcolor[rgb]{0.25,0.44,0.63}{{#1}}}
    \newcommand{\CommentTok}[1]{\textcolor[rgb]{0.38,0.63,0.69}{\textit{{#1}}}}
    \newcommand{\OtherTok}[1]{\textcolor[rgb]{0.00,0.44,0.13}{{#1}}}
    \newcommand{\AlertTok}[1]{\textcolor[rgb]{1.00,0.00,0.00}{\textbf{{#1}}}}
    \newcommand{\FunctionTok}[1]{\textcolor[rgb]{0.02,0.16,0.49}{{#1}}}
    \newcommand{\RegionMarkerTok}[1]{{#1}}
    \newcommand{\ErrorTok}[1]{\textcolor[rgb]{1.00,0.00,0.00}{\textbf{{#1}}}}
    \newcommand{\NormalTok}[1]{{#1}}
    
    % Additional commands for more recent versions of Pandoc
    \newcommand{\ConstantTok}[1]{\textcolor[rgb]{0.53,0.00,0.00}{{#1}}}
    \newcommand{\SpecialCharTok}[1]{\textcolor[rgb]{0.25,0.44,0.63}{{#1}}}
    \newcommand{\VerbatimStringTok}[1]{\textcolor[rgb]{0.25,0.44,0.63}{{#1}}}
    \newcommand{\SpecialStringTok}[1]{\textcolor[rgb]{0.73,0.40,0.53}{{#1}}}
    \newcommand{\ImportTok}[1]{{#1}}
    \newcommand{\DocumentationTok}[1]{\textcolor[rgb]{0.73,0.13,0.13}{\textit{{#1}}}}
    \newcommand{\AnnotationTok}[1]{\textcolor[rgb]{0.38,0.63,0.69}{\textbf{\textit{{#1}}}}}
    \newcommand{\CommentVarTok}[1]{\textcolor[rgb]{0.38,0.63,0.69}{\textbf{\textit{{#1}}}}}
    \newcommand{\VariableTok}[1]{\textcolor[rgb]{0.10,0.09,0.49}{{#1}}}
    \newcommand{\ControlFlowTok}[1]{\textcolor[rgb]{0.00,0.44,0.13}{\textbf{{#1}}}}
    \newcommand{\OperatorTok}[1]{\textcolor[rgb]{0.40,0.40,0.40}{{#1}}}
    \newcommand{\BuiltInTok}[1]{{#1}}
    \newcommand{\ExtensionTok}[1]{{#1}}
    \newcommand{\PreprocessorTok}[1]{\textcolor[rgb]{0.74,0.48,0.00}{{#1}}}
    \newcommand{\AttributeTok}[1]{\textcolor[rgb]{0.49,0.56,0.16}{{#1}}}
    \newcommand{\InformationTok}[1]{\textcolor[rgb]{0.38,0.63,0.69}{\textbf{\textit{{#1}}}}}
    \newcommand{\WarningTok}[1]{\textcolor[rgb]{0.38,0.63,0.69}{\textbf{\textit{{#1}}}}}
    
    
    % Define a nice break command that doesn't care if a line doesn't already
    % exist.
    \def\br{\hspace*{\fill} \\* }
    % Math Jax compatability definitions
    \def\gt{>}
    \def\lt{<}
    % Document parameters
    \title{Introduction to exact real and complex arithmetic with Calcium}
    
    
    

    % Pygments definitions
    
\makeatletter
\def\PY@reset{\let\PY@it=\relax \let\PY@bf=\relax%
    \let\PY@ul=\relax \let\PY@tc=\relax%
    \let\PY@bc=\relax \let\PY@ff=\relax}
\def\PY@tok#1{\csname PY@tok@#1\endcsname}
\def\PY@toks#1+{\ifx\relax#1\empty\else%
    \PY@tok{#1}\expandafter\PY@toks\fi}
\def\PY@do#1{\PY@bc{\PY@tc{\PY@ul{%
    \PY@it{\PY@bf{\PY@ff{#1}}}}}}}
\def\PY#1#2{\PY@reset\PY@toks#1+\relax+\PY@do{#2}}

\expandafter\def\csname PY@tok@w\endcsname{\def\PY@tc##1{\textcolor[rgb]{0.73,0.73,0.73}{##1}}}
\expandafter\def\csname PY@tok@c\endcsname{\let\PY@it=\textit\def\PY@tc##1{\textcolor[rgb]{0.25,0.50,0.50}{##1}}}
\expandafter\def\csname PY@tok@cp\endcsname{\def\PY@tc##1{\textcolor[rgb]{0.74,0.48,0.00}{##1}}}
\expandafter\def\csname PY@tok@k\endcsname{\let\PY@bf=\textbf\def\PY@tc##1{\textcolor[rgb]{0.00,0.50,0.00}{##1}}}
\expandafter\def\csname PY@tok@kp\endcsname{\def\PY@tc##1{\textcolor[rgb]{0.00,0.50,0.00}{##1}}}
\expandafter\def\csname PY@tok@kt\endcsname{\def\PY@tc##1{\textcolor[rgb]{0.69,0.00,0.25}{##1}}}
\expandafter\def\csname PY@tok@o\endcsname{\def\PY@tc##1{\textcolor[rgb]{0.40,0.40,0.40}{##1}}}
\expandafter\def\csname PY@tok@ow\endcsname{\let\PY@bf=\textbf\def\PY@tc##1{\textcolor[rgb]{0.67,0.13,1.00}{##1}}}
\expandafter\def\csname PY@tok@nb\endcsname{\def\PY@tc##1{\textcolor[rgb]{0.00,0.50,0.00}{##1}}}
\expandafter\def\csname PY@tok@nf\endcsname{\def\PY@tc##1{\textcolor[rgb]{0.00,0.00,1.00}{##1}}}
\expandafter\def\csname PY@tok@nc\endcsname{\let\PY@bf=\textbf\def\PY@tc##1{\textcolor[rgb]{0.00,0.00,1.00}{##1}}}
\expandafter\def\csname PY@tok@nn\endcsname{\let\PY@bf=\textbf\def\PY@tc##1{\textcolor[rgb]{0.00,0.00,1.00}{##1}}}
\expandafter\def\csname PY@tok@ne\endcsname{\let\PY@bf=\textbf\def\PY@tc##1{\textcolor[rgb]{0.82,0.25,0.23}{##1}}}
\expandafter\def\csname PY@tok@nv\endcsname{\def\PY@tc##1{\textcolor[rgb]{0.10,0.09,0.49}{##1}}}
\expandafter\def\csname PY@tok@no\endcsname{\def\PY@tc##1{\textcolor[rgb]{0.53,0.00,0.00}{##1}}}
\expandafter\def\csname PY@tok@nl\endcsname{\def\PY@tc##1{\textcolor[rgb]{0.63,0.63,0.00}{##1}}}
\expandafter\def\csname PY@tok@ni\endcsname{\let\PY@bf=\textbf\def\PY@tc##1{\textcolor[rgb]{0.60,0.60,0.60}{##1}}}
\expandafter\def\csname PY@tok@na\endcsname{\def\PY@tc##1{\textcolor[rgb]{0.49,0.56,0.16}{##1}}}
\expandafter\def\csname PY@tok@nt\endcsname{\let\PY@bf=\textbf\def\PY@tc##1{\textcolor[rgb]{0.00,0.50,0.00}{##1}}}
\expandafter\def\csname PY@tok@nd\endcsname{\def\PY@tc##1{\textcolor[rgb]{0.67,0.13,1.00}{##1}}}
\expandafter\def\csname PY@tok@s\endcsname{\def\PY@tc##1{\textcolor[rgb]{0.73,0.13,0.13}{##1}}}
\expandafter\def\csname PY@tok@sd\endcsname{\let\PY@it=\textit\def\PY@tc##1{\textcolor[rgb]{0.73,0.13,0.13}{##1}}}
\expandafter\def\csname PY@tok@si\endcsname{\let\PY@bf=\textbf\def\PY@tc##1{\textcolor[rgb]{0.73,0.40,0.53}{##1}}}
\expandafter\def\csname PY@tok@se\endcsname{\let\PY@bf=\textbf\def\PY@tc##1{\textcolor[rgb]{0.73,0.40,0.13}{##1}}}
\expandafter\def\csname PY@tok@sr\endcsname{\def\PY@tc##1{\textcolor[rgb]{0.73,0.40,0.53}{##1}}}
\expandafter\def\csname PY@tok@ss\endcsname{\def\PY@tc##1{\textcolor[rgb]{0.10,0.09,0.49}{##1}}}
\expandafter\def\csname PY@tok@sx\endcsname{\def\PY@tc##1{\textcolor[rgb]{0.00,0.50,0.00}{##1}}}
\expandafter\def\csname PY@tok@m\endcsname{\def\PY@tc##1{\textcolor[rgb]{0.40,0.40,0.40}{##1}}}
\expandafter\def\csname PY@tok@gh\endcsname{\let\PY@bf=\textbf\def\PY@tc##1{\textcolor[rgb]{0.00,0.00,0.50}{##1}}}
\expandafter\def\csname PY@tok@gu\endcsname{\let\PY@bf=\textbf\def\PY@tc##1{\textcolor[rgb]{0.50,0.00,0.50}{##1}}}
\expandafter\def\csname PY@tok@gd\endcsname{\def\PY@tc##1{\textcolor[rgb]{0.63,0.00,0.00}{##1}}}
\expandafter\def\csname PY@tok@gi\endcsname{\def\PY@tc##1{\textcolor[rgb]{0.00,0.63,0.00}{##1}}}
\expandafter\def\csname PY@tok@gr\endcsname{\def\PY@tc##1{\textcolor[rgb]{1.00,0.00,0.00}{##1}}}
\expandafter\def\csname PY@tok@ge\endcsname{\let\PY@it=\textit}
\expandafter\def\csname PY@tok@gs\endcsname{\let\PY@bf=\textbf}
\expandafter\def\csname PY@tok@gp\endcsname{\let\PY@bf=\textbf\def\PY@tc##1{\textcolor[rgb]{0.00,0.00,0.50}{##1}}}
\expandafter\def\csname PY@tok@go\endcsname{\def\PY@tc##1{\textcolor[rgb]{0.53,0.53,0.53}{##1}}}
\expandafter\def\csname PY@tok@gt\endcsname{\def\PY@tc##1{\textcolor[rgb]{0.00,0.27,0.87}{##1}}}
\expandafter\def\csname PY@tok@err\endcsname{\def\PY@bc##1{\setlength{\fboxsep}{0pt}\fcolorbox[rgb]{1.00,0.00,0.00}{1,1,1}{\strut ##1}}}
\expandafter\def\csname PY@tok@kc\endcsname{\let\PY@bf=\textbf\def\PY@tc##1{\textcolor[rgb]{0.00,0.50,0.00}{##1}}}
\expandafter\def\csname PY@tok@kd\endcsname{\let\PY@bf=\textbf\def\PY@tc##1{\textcolor[rgb]{0.00,0.50,0.00}{##1}}}
\expandafter\def\csname PY@tok@kn\endcsname{\let\PY@bf=\textbf\def\PY@tc##1{\textcolor[rgb]{0.00,0.50,0.00}{##1}}}
\expandafter\def\csname PY@tok@kr\endcsname{\let\PY@bf=\textbf\def\PY@tc##1{\textcolor[rgb]{0.00,0.50,0.00}{##1}}}
\expandafter\def\csname PY@tok@bp\endcsname{\def\PY@tc##1{\textcolor[rgb]{0.00,0.50,0.00}{##1}}}
\expandafter\def\csname PY@tok@fm\endcsname{\def\PY@tc##1{\textcolor[rgb]{0.00,0.00,1.00}{##1}}}
\expandafter\def\csname PY@tok@vc\endcsname{\def\PY@tc##1{\textcolor[rgb]{0.10,0.09,0.49}{##1}}}
\expandafter\def\csname PY@tok@vg\endcsname{\def\PY@tc##1{\textcolor[rgb]{0.10,0.09,0.49}{##1}}}
\expandafter\def\csname PY@tok@vi\endcsname{\def\PY@tc##1{\textcolor[rgb]{0.10,0.09,0.49}{##1}}}
\expandafter\def\csname PY@tok@vm\endcsname{\def\PY@tc##1{\textcolor[rgb]{0.10,0.09,0.49}{##1}}}
\expandafter\def\csname PY@tok@sa\endcsname{\def\PY@tc##1{\textcolor[rgb]{0.73,0.13,0.13}{##1}}}
\expandafter\def\csname PY@tok@sb\endcsname{\def\PY@tc##1{\textcolor[rgb]{0.73,0.13,0.13}{##1}}}
\expandafter\def\csname PY@tok@sc\endcsname{\def\PY@tc##1{\textcolor[rgb]{0.73,0.13,0.13}{##1}}}
\expandafter\def\csname PY@tok@dl\endcsname{\def\PY@tc##1{\textcolor[rgb]{0.73,0.13,0.13}{##1}}}
\expandafter\def\csname PY@tok@s2\endcsname{\def\PY@tc##1{\textcolor[rgb]{0.73,0.13,0.13}{##1}}}
\expandafter\def\csname PY@tok@sh\endcsname{\def\PY@tc##1{\textcolor[rgb]{0.73,0.13,0.13}{##1}}}
\expandafter\def\csname PY@tok@s1\endcsname{\def\PY@tc##1{\textcolor[rgb]{0.73,0.13,0.13}{##1}}}
\expandafter\def\csname PY@tok@mb\endcsname{\def\PY@tc##1{\textcolor[rgb]{0.40,0.40,0.40}{##1}}}
\expandafter\def\csname PY@tok@mf\endcsname{\def\PY@tc##1{\textcolor[rgb]{0.40,0.40,0.40}{##1}}}
\expandafter\def\csname PY@tok@mh\endcsname{\def\PY@tc##1{\textcolor[rgb]{0.40,0.40,0.40}{##1}}}
\expandafter\def\csname PY@tok@mi\endcsname{\def\PY@tc##1{\textcolor[rgb]{0.40,0.40,0.40}{##1}}}
\expandafter\def\csname PY@tok@il\endcsname{\def\PY@tc##1{\textcolor[rgb]{0.40,0.40,0.40}{##1}}}
\expandafter\def\csname PY@tok@mo\endcsname{\def\PY@tc##1{\textcolor[rgb]{0.40,0.40,0.40}{##1}}}
\expandafter\def\csname PY@tok@ch\endcsname{\let\PY@it=\textit\def\PY@tc##1{\textcolor[rgb]{0.25,0.50,0.50}{##1}}}
\expandafter\def\csname PY@tok@cm\endcsname{\let\PY@it=\textit\def\PY@tc##1{\textcolor[rgb]{0.25,0.50,0.50}{##1}}}
\expandafter\def\csname PY@tok@cpf\endcsname{\let\PY@it=\textit\def\PY@tc##1{\textcolor[rgb]{0.25,0.50,0.50}{##1}}}
\expandafter\def\csname PY@tok@c1\endcsname{\let\PY@it=\textit\def\PY@tc##1{\textcolor[rgb]{0.25,0.50,0.50}{##1}}}
\expandafter\def\csname PY@tok@cs\endcsname{\let\PY@it=\textit\def\PY@tc##1{\textcolor[rgb]{0.25,0.50,0.50}{##1}}}

\def\PYZbs{\char`\\}
\def\PYZus{\char`\_}
\def\PYZob{\char`\{}
\def\PYZcb{\char`\}}
\def\PYZca{\char`\^}
\def\PYZam{\char`\&}
\def\PYZlt{\char`\<}
\def\PYZgt{\char`\>}
\def\PYZsh{\char`\#}
\def\PYZpc{\char`\%}
\def\PYZdl{\char`\$}
\def\PYZhy{\char`\-}
\def\PYZsq{\char`\'}
\def\PYZdq{\char`\"}
\def\PYZti{\char`\~}
% for compatibility with earlier versions
\def\PYZat{@}
\def\PYZlb{[}
\def\PYZrb{]}
\makeatother


    % Exact colors from NB
    \definecolor{incolor}{rgb}{0.0, 0.0, 0.5}
    \definecolor{outcolor}{rgb}{0.545, 0.0, 0.0}



    
    % Prevent overflowing lines due to hard-to-break entities
    \sloppy 
    % Setup hyperref package
    \hypersetup{
      breaklinks=true,  % so long urls are correctly broken across lines
      colorlinks=true,
      urlcolor=urlcolor,
      linkcolor=linkcolor,
      citecolor=citecolor,
      }
    % Slightly bigger margins than the latex defaults
    
    \geometry{verbose,tmargin=1in,bmargin=1in,lmargin=1in,rmargin=1in}
    
    

    \begin{document}
    
    
    \maketitle
    
    

    
    \section{Introduction to exact real and complex arithmetic with
Calcium}\label{introduction-to-exact-real-and-complex-arithmetic-with-calcium}

    This notebook gives an introduction Calcium
(https://fredrikj.net/calcium/), using the builtin ctypes-based Python
wrapper.

\emph{Warning}: the ctypes wrapper is incomplete, slow, and has some
problems (e.g. memory leaks). It is mainly intended for testing the
library. A more robust Cython interface should eventually be developed
for "production use", along with a Julia interface. Calcium itself is
also still experimental.

    \subsection{Setup}\label{setup}

To run this Notebook locally, you need to build Calcium and add
\texttt{calcium/pycalcium} to your Python path.

    \begin{Verbatim}[commandchars=\\\{\}]
{\color{incolor}In [{\color{incolor}1}]:} \PY{c+c1}{\PYZsh{} For convenience}
        \PY{k+kn}{from} \PY{n+nn}{IPython}\PY{n+nn}{.}\PY{n+nn}{display} \PY{k+kn}{import} \PY{n}{display}
        
        \PY{c+c1}{\PYZsh{} Import the Calcium ctypes wrapper}
        \PY{c+c1}{\PYZsh{} Don\PYZsq{}t \PYZdq{}import *\PYZdq{} at home \PYZhy{}\PYZhy{} this is for demonstration use only}
        \PY{k+kn}{from} \PY{n+nn}{pyca} \PY{k+kn}{import} \PY{o}{*}
\end{Verbatim}


    Quick example: Euler's identity \(e^{\pi i} + 1 = 0\)

    \begin{Verbatim}[commandchars=\\\{\}]
{\color{incolor}In [{\color{incolor}2}]:} \PY{n}{exp}\PY{p}{(}\PY{n}{pi} \PY{o}{*} \PY{n}{i}\PY{p}{)} \PY{o}{+} \PY{l+m+mi}{1}
\end{Verbatim}

\texttt{\color{outcolor}Out[{\color{outcolor}2}]:}
    
    $$0$$

    

    \subsection{Basic types}\label{basic-types}

    The core types implemented in Calcium are the following:

\begin{itemize}
\tightlist
\item
  Symbolic expressions - \texttt{pyca.fexpr} (\texttt{fexpr\_t} in C)
\item
  Algebraic numbers - \texttt{pyca.qqbar} (\texttt{qqbar\_t} in C)
\item
  Calcium numbers / field elements - \texttt{pyca.ca} (\texttt{ca\_t} in
  C)
\end{itemize}

The types have different roles, and it is important to understand their
differences.

\textbf{Symbolic expressions} are the most flexible representation: they
preserve the exact form of the input without performing any computation
or simplification. For example, \(\sqrt{2} / 2\), \(1 / \sqrt{2}\) and
\(2^{1/2} / 2\) are all different as expressions (although they
represent the same number). Symbolic expressions are not very useful on
their own for computations, but they are convenient as input and output
for other, more "computational" types.

\textbf{Algebraic numbers} represent elements of the field
\(\overline{\mathbb{Q}}\) in \emph{canonical form}, consisting of the
number's minimal polynomial together with an isolating complex interval
for a unique root. Thus one number can only have one representation:
\(\sqrt{2} / 2\) and \(1 / \sqrt{2}\) will evaluate to exactly the same
\texttt{qqbar} algebraic number. This is useful because the results are
predictable and in a sense maximally simplified (contrary to the other
types, it is impossible to construct a complicated \texttt{qqbar}
instance that represents 0 without trivially being 0). The downsides are
that field operations are much more expensive than in the \texttt{ca}
representation, and in passing from a symbolic expression to a
\texttt{qqbar}, structural information (such as a simple closed form)
may be lost.

\textbf{Calcium numbers} represent numbers as elements of fields
\(\mathbb{Q}(a_1,\ldots,a_n)\). Calcium constructs extension numbers
\(a_k\) and fields \(\mathbb{Q}(a_1,\ldots,a_n)\) automatically and
lazily, so from the point of view of the user Calcium numbers behave
just like numbers. The extension numbers \(a_k\) can be "absolute"
algebraic numbers (\texttt{qqbar} instances), but they can also be
transcendental numbers like \(\pi\) or \(\exp(\sqrt{2} i)\). This
representation is highly efficient for arithmetic, but in general does
not guarantee a canonical form. Relations between extension elements
(e.g. \(\log(4) / \log(2) = 2\)) are simplified automatically using
ideal reduction, but Calcium will not be able to prove all relations, so
it is possible to have a \texttt{ca} instance that represents 0 without
trivially being 0 (such instances are nevertheless handled in a
mathematically rigorous way as discussed below). The \texttt{ca} type
can also represent special values and metavalues (signed and unsigned
infinities, undefined, unknown).

    \subsection{Symbolic expressions}\label{symbolic-expressions}

In the \texttt{pyca} namespace, lowercase names (example: \texttt{sqrt})
denote \texttt{ca} functions while while uppercase names (example:
\texttt{Sqrt}) denote \texttt{fexpr} symbolic expressions. It is easy to
construct symbolic expressions:

    \begin{Verbatim}[commandchars=\\\{\}]
{\color{incolor}In [{\color{incolor}3}]:} \PY{c+c1}{\PYZsh{} Create a symbolic sum of three terms}
        \PY{n}{expr} \PY{o}{=} \PY{n}{Add}\PY{p}{(}\PY{n}{Sqrt}\PY{p}{(}\PY{l+m+mi}{2}\PY{p}{)} \PY{o}{/} \PY{l+m+mi}{2}\PY{p}{,} \PY{l+m+mi}{1} \PY{o}{/} \PY{n}{Sqrt}\PY{p}{(}\PY{l+m+mi}{2}\PY{p}{)}\PY{p}{,} \PY{l+m+mi}{2} \PY{o}{*}\PY{o}{*} \PY{n}{Div}\PY{p}{(}\PY{l+m+mi}{1}\PY{p}{,} \PY{l+m+mi}{2}\PY{p}{)} \PY{o}{/} \PY{l+m+mi}{2}\PY{p}{)}
        \PY{c+c1}{\PYZsh{} Display the formula}
        \PY{n}{expr}
\end{Verbatim}

\texttt{\color{outcolor}Out[{\color{outcolor}3}]:}
    
    $$\frac{\sqrt{2}}{2} + \frac{1}{\sqrt{2}} + \frac{{2}^{1 / 2}}{2}$$

    

    \begin{Verbatim}[commandchars=\\\{\}]
{\color{incolor}In [{\color{incolor}4}]:} \PY{n+nb}{type}\PY{p}{(}\PY{n}{expr}\PY{p}{)}
\end{Verbatim}


\begin{Verbatim}[commandchars=\\\{\}]
{\color{outcolor}Out[{\color{outcolor}4}]:} pyca.fexpr
\end{Verbatim}
            
    \subsubsection{Evaluating expressions and displaying
objects}\label{evaluating-expressions-and-displaying-objects}

    Constant expressions can be passed as input to other types, resulting in
evaluation:

    \begin{Verbatim}[commandchars=\\\{\}]
{\color{incolor}In [{\color{incolor}5}]:} \PY{n}{ca}\PY{p}{(}\PY{n}{expr}\PY{p}{)}     \PY{c+c1}{\PYZsh{} evaluate expr using ca arithmetic, producing a ca}
\end{Verbatim}

\texttt{\color{outcolor}Out[{\color{outcolor}5}]:}
    
    $$\frac{3 a_{1}}{2}\; \text{ where } a_{1} = \sqrt{2}$$

    

    \begin{Verbatim}[commandchars=\\\{\}]
{\color{incolor}In [{\color{incolor}6}]:} \PY{n+nb}{type}\PY{p}{(}\PY{n}{\PYZus{}}\PY{p}{)}
\end{Verbatim}


\begin{Verbatim}[commandchars=\\\{\}]
{\color{outcolor}Out[{\color{outcolor}6}]:} pyca.ca
\end{Verbatim}
            
    \begin{Verbatim}[commandchars=\\\{\}]
{\color{incolor}In [{\color{incolor}7}]:} \PY{n}{qqbar}\PY{p}{(}\PY{n}{expr}\PY{p}{)}     \PY{c+c1}{\PYZsh{} evaluate expr using qqbar arithmetic, producing a qqbar}
\end{Verbatim}

\texttt{\color{outcolor}Out[{\color{outcolor}7}]:}
    
    $$\left(\text{Root }\, x \approx {2.12132} \;\text{ of } \;{2 x^{2}-9}\right)$$

    

    \begin{Verbatim}[commandchars=\\\{\}]
{\color{incolor}In [{\color{incolor}8}]:} \PY{n+nb}{type}\PY{p}{(}\PY{n}{\PYZus{}}\PY{p}{)}
\end{Verbatim}


\begin{Verbatim}[commandchars=\\\{\}]
{\color{outcolor}Out[{\color{outcolor}8}]:} pyca.qqbar
\end{Verbatim}
            
    Symbolic expressions are also generated automatically behind the scenes
in order to display objects in LaTeX in the notebook (this is done in
the examples above). You can of course also print objects in text form:

    \begin{Verbatim}[commandchars=\\\{\}]
{\color{incolor}In [{\color{incolor}9}]:} \PY{n+nb}{print}\PY{p}{(}\PY{n}{expr}\PY{p}{)}
        \PY{n+nb}{print}\PY{p}{(}\PY{n}{ca}\PY{p}{(}\PY{n}{expr}\PY{p}{)}\PY{p}{)}
        \PY{n+nb}{print}\PY{p}{(}\PY{n}{qqbar}\PY{p}{(}\PY{n}{expr}\PY{p}{)}\PY{p}{)}
\end{Verbatim}


    \begin{Verbatim}[commandchars=\\\{\}]
Add(Div(Sqrt(2), 2), Div(1, Sqrt(2)), Div(Pow(2, Div(1, 2)), 2))
2.12132 \{(3*a)/2 where a = 1.41421 [a\^{}2-2=0]\}
2.12132 (deg 2)

    \end{Verbatim}

    By default, \texttt{qqbar} objects display as polynomial roots. To
produce a closed-form expression (if possible), the
\texttt{qqbar.fexpr()} method can be called:

    \begin{Verbatim}[commandchars=\\\{\}]
{\color{incolor}In [{\color{incolor}10}]:} \PY{n}{qqbar}\PY{p}{(}\PY{n}{expr}\PY{p}{)}\PY{o}{.}\PY{n}{fexpr}\PY{p}{(}\PY{p}{)}
\end{Verbatim}

\texttt{\color{outcolor}Out[{\color{outcolor}10}]:}
    
    $$\frac{3 \sqrt{2}}{2}$$

    

    \subsubsection{Manipulating symbolic
expressions}\label{manipulating-symbolic-expressions}

The \texttt{fexpr} type provides methods for rudimentary manipulation
(accessing subexpressions and so on).

    \begin{Verbatim}[commandchars=\\\{\}]
{\color{incolor}In [{\color{incolor}11}]:} \PY{n}{expr} \PY{o}{=} \PY{n}{Add}\PY{p}{(}\PY{n}{Sqrt}\PY{p}{(}\PY{l+m+mi}{2}\PY{p}{)} \PY{o}{/} \PY{l+m+mi}{2}\PY{p}{,} \PY{l+m+mi}{1} \PY{o}{/} \PY{n}{Sqrt}\PY{p}{(}\PY{l+m+mi}{2}\PY{p}{)}\PY{p}{,} \PY{l+m+mi}{2} \PY{o}{*}\PY{o}{*} \PY{n}{Div}\PY{p}{(}\PY{l+m+mi}{1}\PY{p}{,} \PY{l+m+mi}{2}\PY{p}{)} \PY{o}{/} \PY{l+m+mi}{2}\PY{p}{)}
         \PY{n+nb}{print}\PY{p}{(}\PY{n}{expr}\PY{o}{.}\PY{n}{head}\PY{p}{(}\PY{p}{)}\PY{p}{)}
         \PY{n+nb}{print}\PY{p}{(}\PY{n}{expr}\PY{o}{.}\PY{n}{nargs}\PY{p}{(}\PY{p}{)}\PY{p}{)}
         \PY{n+nb}{print}\PY{p}{(}\PY{n}{expr}\PY{o}{.}\PY{n}{args}\PY{p}{(}\PY{p}{)}\PY{p}{)}
         \PY{n+nb}{print}\PY{p}{(}\PY{n}{expr}\PY{o}{.}\PY{n}{is\PYZus{}atom}\PY{p}{(}\PY{p}{)}\PY{p}{)}
         \PY{n+nb}{print}\PY{p}{(}\PY{n}{expr}\PY{o}{.}\PY{n}{args}\PY{p}{(}\PY{p}{)}\PY{p}{[}\PY{l+m+mi}{0}\PY{p}{]}\PY{o}{.}\PY{n}{args}\PY{p}{(}\PY{p}{)}\PY{p}{[}\PY{l+m+mi}{1}\PY{p}{]}\PY{o}{.}\PY{n}{is\PYZus{}atom}\PY{p}{(}\PY{p}{)}\PY{p}{)}
         \PY{n+nb}{print}\PY{p}{(}\PY{n}{expr}\PY{o}{.}\PY{n}{num\PYZus{}leaves}\PY{p}{(}\PY{p}{)}\PY{p}{)}
         \PY{n+nb}{print}\PY{p}{(}\PY{n}{expr}\PY{o}{.}\PY{n}{nwords}\PY{p}{(}\PY{p}{)}\PY{p}{)}
         \PY{n+nb}{print}\PY{p}{(}\PY{n}{expr}\PY{o}{.}\PY{n}{replace}\PY{p}{(}\PY{l+m+mi}{2}\PY{p}{,} \PY{n}{Pi}\PY{p}{)}\PY{p}{)}
\end{Verbatim}


    \begin{Verbatim}[commandchars=\\\{\}]
Add
3
(Div(Sqrt(2), 2), Div(1, Sqrt(2)), Div(Pow(2, Div(1, 2)), 2))
False
True
16
24
Add(Div(Sqrt(Pi), Pi), Div(1, Sqrt(Pi)), Div(Pow(Pi, Div(1, Pi)), Pi))

    \end{Verbatim}

    The \texttt{.nstr()} method computes a numerical approximation using
Arb, returning a decimal string:

    \begin{Verbatim}[commandchars=\\\{\}]
{\color{incolor}In [{\color{incolor}12}]:} \PY{n+nb}{print}\PY{p}{(}\PY{p}{(}\PY{n}{Sqrt}\PY{p}{(}\PY{l+m+mi}{2}\PY{p}{)} \PY{o}{/} \PY{l+m+mi}{2}\PY{p}{)}\PY{o}{.}\PY{n}{nstr}\PY{p}{(}\PY{l+m+mi}{30}\PY{p}{)}\PY{p}{)}
         \PY{n+nb}{print}\PY{p}{(}\PY{p}{(}\PY{n}{Exp}\PY{p}{(}\PY{l+m+mi}{1} \PY{o}{+} \PY{l+m+mi}{2}\PY{n}{j}\PY{p}{)}\PY{o}{.}\PY{n}{nstr}\PY{p}{(}\PY{l+m+mi}{10}\PY{p}{)}\PY{p}{)}\PY{p}{)}
         
         \PY{c+c1}{\PYZsh{} No symbolic simplification is done \PYZhy{} Arb can generally not detect}
         \PY{c+c1}{\PYZsh{} exact zeros, and zeros will be output in the form 0e\PYZhy{}n}
         \PY{n+nb}{print}\PY{p}{(}\PY{p}{(}\PY{n}{Exp}\PY{p}{(}\PY{n}{Pi}\PY{o}{*}\PY{n}{NumberI}\PY{p}{)}\PY{p}{)}\PY{o}{.}\PY{n}{nstr}\PY{p}{(}\PY{l+m+mi}{30}\PY{p}{)}\PY{p}{)}
         \PY{n+nb}{print}\PY{p}{(}\PY{p}{(}\PY{n}{Sqrt}\PY{p}{(}\PY{l+m+mi}{2}\PY{p}{)}\PY{o}{/}\PY{l+m+mi}{2} \PY{o}{\PYZhy{}} \PY{l+m+mi}{1}\PY{o}{/}\PY{n}{Sqrt}\PY{p}{(}\PY{l+m+mi}{2}\PY{p}{)}\PY{p}{)}\PY{o}{.}\PY{n}{nstr}\PY{p}{(}\PY{p}{)}\PY{p}{)}
\end{Verbatim}


    \begin{Verbatim}[commandchars=\\\{\}]
0.707106781186547524400844362105
-1.131204384 + 2.471726672*I
-1.00000000000000000000000000000 + 0e-37*I
0e-731

    \end{Verbatim}

    The \texttt{.expanded\_normal\_form()} method puts the given formula in
a canonical form as a formal rational expression.

    \begin{Verbatim}[commandchars=\\\{\}]
{\color{incolor}In [{\color{incolor}13}]:} \PY{n}{x} \PY{o}{=} \PY{n}{fexpr}\PY{p}{(}\PY{l+s+s2}{\PYZdq{}}\PY{l+s+s2}{x}\PY{l+s+s2}{\PYZdq{}}\PY{p}{)}\PY{p}{;} \PY{n}{y} \PY{o}{=} \PY{n}{fexpr}\PY{p}{(}\PY{l+s+s2}{\PYZdq{}}\PY{l+s+s2}{y}\PY{l+s+s2}{\PYZdq{}}\PY{p}{)}
         \PY{n}{A} \PY{o}{=} \PY{p}{(}\PY{n}{x}\PY{o}{+}\PY{n}{y}\PY{p}{)}\PY{o}{*}\PY{o}{*}\PY{l+m+mi}{5} \PY{o}{*} \PY{p}{(}\PY{n}{x}\PY{o}{\PYZhy{}}\PY{n}{y}\PY{p}{)} \PY{o}{*} \PY{p}{(}\PY{n}{x} \PY{o}{+} \PY{l+m+mi}{1}\PY{p}{)}
         \PY{n}{B} \PY{o}{=} \PY{p}{(}\PY{n}{x}\PY{o}{*}\PY{o}{*}\PY{l+m+mi}{2} \PY{o}{\PYZhy{}} \PY{n}{y}\PY{o}{*}\PY{o}{*}\PY{l+m+mi}{2}\PY{p}{)} \PY{o}{*} \PY{p}{(}\PY{n}{x}\PY{o}{*}\PY{o}{*}\PY{l+m+mi}{2} \PY{o}{\PYZhy{}} \PY{l+m+mi}{1}\PY{p}{)}
         \PY{p}{(}\PY{n}{A} \PY{o}{/} \PY{n}{B}\PY{p}{)}\PY{o}{.}\PY{n}{expanded\PYZus{}normal\PYZus{}form}\PY{p}{(}\PY{p}{)}
\end{Verbatim}

\texttt{\color{outcolor}Out[{\color{outcolor}13}]:}
    
    $$\frac{{x}^{4} + 4 {x}^{3} y + 6 {x}^{2} {y}^{2} + 4 x {y}^{3} + {y}^{4}}{x-1}$$

    

    Please note that \texttt{.expanded\_normal\_form()} only simplifies
rational arithmetic operations, treating anything non-arithmetical as an
atomic node. For example, square roots are treated as atomic. It also
does not simplify nodes recursively.

    \begin{Verbatim}[commandchars=\\\{\}]
{\color{incolor}In [{\color{incolor}14}]:} \PY{n}{display}\PY{p}{(}\PY{p}{(}\PY{n}{Sqrt}\PY{p}{(}\PY{n}{x}\PY{p}{)} \PY{o}{/} \PY{n}{Sqrt}\PY{p}{(}\PY{n}{x}\PY{p}{)}\PY{p}{)}\PY{o}{.}\PY{n}{expanded\PYZus{}normal\PYZus{}form}\PY{p}{(}\PY{p}{)}\PY{p}{)}
         \PY{n}{display}\PY{p}{(}\PY{p}{(}\PY{n}{Sqrt}\PY{p}{(}\PY{n}{x}\PY{p}{)} \PY{o}{*} \PY{n}{Sqrt}\PY{p}{(}\PY{n}{x}\PY{p}{)}\PY{p}{)}\PY{o}{.}\PY{n}{expanded\PYZus{}normal\PYZus{}form}\PY{p}{(}\PY{p}{)}\PY{p}{)}
         \PY{n}{display}\PY{p}{(}\PY{p}{(}\PY{n}{Sqrt}\PY{p}{(}\PY{l+m+mi}{2}\PY{o}{*}\PY{n}{x}\PY{p}{)} \PY{o}{/} \PY{n}{Sqrt}\PY{p}{(}\PY{n}{x}\PY{o}{*}\PY{l+m+mi}{2}\PY{p}{)}\PY{p}{)}\PY{o}{.}\PY{n}{expanded\PYZus{}normal\PYZus{}form}\PY{p}{(}\PY{p}{)}\PY{p}{)}
\end{Verbatim}


    $$1$$

    
    $${\left(\sqrt{x}\right)}^{2}$$

    
    $$\frac{\sqrt{2 x}}{\sqrt{x \cdot 2}}$$

    
    We will not do anything more sophisticated with symbolic expressions in
this notebook; we will instead move on to describing exact numerical
calculations using the \texttt{ca} and \texttt{qqbar} types.

    \subsection{Calcium numbers}\label{calcium-numbers}

Calcium numbers encompass rational numbers, of course:

    \begin{Verbatim}[commandchars=\\\{\}]
{\color{incolor}In [{\color{incolor}15}]:} \PY{n}{ca}\PY{p}{(}\PY{l+m+mi}{1}\PY{p}{)} \PY{o}{/} \PY{l+m+mi}{3}
\end{Verbatim}

\texttt{\color{outcolor}Out[{\color{outcolor}15}]:}
    
    $$\frac{1}{3}$$

    

    \begin{Verbatim}[commandchars=\\\{\}]
{\color{incolor}In [{\color{incolor}16}]:} \PY{p}{(}\PY{n}{ca}\PY{p}{(}\PY{l+m+mi}{1}\PY{p}{)} \PY{o}{/} \PY{l+m+mi}{3}\PY{p}{)} \PY{o}{*} \PY{l+m+mi}{6}
\end{Verbatim}

\texttt{\color{outcolor}Out[{\color{outcolor}16}]:}
    
    $$2$$

    

    Irrational numbers result in extensions of \(\mathbb{Q}\):

    \begin{Verbatim}[commandchars=\\\{\}]
{\color{incolor}In [{\color{incolor}17}]:} \PY{c+c1}{\PYZsh{} Alternative syntax: ca(2).sqrt()}
         \PY{n}{sqrt}\PY{p}{(}\PY{l+m+mi}{2}\PY{p}{)}
\end{Verbatim}

\texttt{\color{outcolor}Out[{\color{outcolor}17}]:}
    
    $$a_{1}\; \text{ where } a_{1} = \sqrt{2}$$

    

    \begin{Verbatim}[commandchars=\\\{\}]
{\color{incolor}In [{\color{incolor}18}]:} \PY{n}{sqrt}\PY{p}{(}\PY{l+m+mi}{2}\PY{p}{)} \PY{o}{*}\PY{o}{*} \PY{l+m+mi}{2}
\end{Verbatim}

\texttt{\color{outcolor}Out[{\color{outcolor}18}]:}
    
    $$2$$

    

    \begin{Verbatim}[commandchars=\\\{\}]
{\color{incolor}In [{\color{incolor}19}]:} \PY{l+m+mi}{2} \PY{o}{*} \PY{n}{pi}
\end{Verbatim}

\texttt{\color{outcolor}Out[{\color{outcolor}19}]:}
    
    $$2 a_{1}\; \text{ where } a_{1} = \pi$$

    

    Field arithmetic produces numbers represented as formal fraction field
elements:

    \begin{Verbatim}[commandchars=\\\{\}]
{\color{incolor}In [{\color{incolor}20}]:} \PY{n}{pi} \PY{o}{*} \PY{n}{i} \PY{o}{*} \PY{n}{sqrt}\PY{p}{(}\PY{l+m+mi}{2}\PY{p}{)}
\end{Verbatim}

\texttt{\color{outcolor}Out[{\color{outcolor}20}]:}
    
    $$a_{1} a_{2} a_{3}\; \text{ where } a_{1} = \pi,\;a_{2} = \sqrt{2},\;a_{3} = i$$

    

    \begin{Verbatim}[commandchars=\\\{\}]
{\color{incolor}In [{\color{incolor}21}]:} \PY{p}{(}\PY{n}{pi} \PY{o}{*} \PY{n}{i} \PY{o}{*} \PY{n}{sqrt}\PY{p}{(}\PY{l+m+mi}{2}\PY{p}{)}\PY{p}{)} \PY{o}{*}\PY{o}{*} \PY{l+m+mi}{2}   \PY{c+c1}{\PYZsh{} note the simplifications}
\end{Verbatim}

\texttt{\color{outcolor}Out[{\color{outcolor}21}]:}
    
    $$-2 a^{2}_{1}\; \text{ where } a_{1} = \pi,\;a_{2} = \sqrt{2},\;a_{3} = i$$

    

    \begin{Verbatim}[commandchars=\\\{\}]
{\color{incolor}In [{\color{incolor}22}]:} \PY{p}{(}\PY{p}{(}\PY{n}{pi} \PY{o}{+} \PY{n}{i} \PY{o}{+} \PY{n}{sqrt}\PY{p}{(}\PY{l+m+mi}{2}\PY{p}{)}\PY{p}{)} \PY{o}{/} \PY{p}{(}\PY{n}{pi} \PY{o}{+} \PY{n}{sqrt}\PY{p}{(}\PY{l+m+mi}{2}\PY{p}{)}\PY{p}{)}\PY{p}{)}\PY{o}{*}\PY{o}{*}\PY{l+m+mi}{3}
\end{Verbatim}

\texttt{\color{outcolor}Out[{\color{outcolor}22}]:}
    
    $$\frac{a^{3}_{1} + 3 a^{2}_{1} a_{2} + 3 a^{2}_{1} a_{3} + 6 a_{1} a_{2} a_{3} + 3 a_{1}- a_{2} + 5 a_{3}}{a^{3}_{1} + 3 a^{2}_{1} a_{2} + 6 a_{1} + 2 a_{2}}\; \text{ where } a_{1} = \pi,\;a_{2} = \sqrt{2},\;a_{3} = i$$

    

    \subsubsection{Some more number field
arithmetic}\label{some-more-number-field-arithmetic}

    Let us construct the golden ratio:

    \begin{Verbatim}[commandchars=\\\{\}]
{\color{incolor}In [{\color{incolor}23}]:} \PY{n}{phi} \PY{o}{=} \PY{p}{(}\PY{n}{sqrt}\PY{p}{(}\PY{l+m+mi}{5}\PY{p}{)} \PY{o}{+} \PY{l+m+mi}{1}\PY{p}{)} \PY{o}{/} \PY{l+m+mi}{2}
         \PY{n}{phi}
\end{Verbatim}

\texttt{\color{outcolor}Out[{\color{outcolor}23}]:}
    
    $$\frac{a_{1} + 1}{2}\; \text{ where } a_{1} = \sqrt{5}$$

    

    We compute the 200th Fibonacci number using Binet's formula:

    \begin{Verbatim}[commandchars=\\\{\}]
{\color{incolor}In [{\color{incolor}24}]:} \PY{p}{(}\PY{n}{phi}\PY{o}{*}\PY{o}{*}\PY{l+m+mi}{200} \PY{o}{\PYZhy{}} \PY{p}{(}\PY{l+m+mi}{1}\PY{o}{\PYZhy{}}\PY{n}{phi}\PY{p}{)}\PY{o}{*}\PY{o}{*}\PY{l+m+mi}{200}\PY{p}{)} \PY{o}{/} \PY{n}{sqrt}\PY{p}{(}\PY{l+m+mi}{5}\PY{p}{)}
\end{Verbatim}

\texttt{\color{outcolor}Out[{\color{outcolor}24}]:}
    
    $$280571172992510140037611932413038677189525$$

    

    Depending on the operations, \texttt{ca} arithmetic may result in
different field representations of the same number:

    \begin{Verbatim}[commandchars=\\\{\}]
{\color{incolor}In [{\color{incolor}25}]:} \PY{n}{display}\PY{p}{(}\PY{n}{sqrt}\PY{p}{(}\PY{l+m+mi}{2}\PY{p}{)}\PY{o}{*}\PY{n}{sqrt}\PY{p}{(}\PY{l+m+mi}{3}\PY{p}{)}\PY{p}{)}
         \PY{n}{display}\PY{p}{(}\PY{n}{sqrt}\PY{p}{(}\PY{l+m+mi}{6}\PY{p}{)}\PY{p}{)}
\end{Verbatim}


    $$a_{1} a_{2}\; \text{ where } a_{1} = \sqrt{3},\;a_{2} = \sqrt{2}$$

    
    $$a_{1}\; \text{ where } a_{1} = \sqrt{6}$$

    
    The difference simplifies to zero:

    \begin{Verbatim}[commandchars=\\\{\}]
{\color{incolor}In [{\color{incolor}26}]:} \PY{n}{display}\PY{p}{(}\PY{n}{sqrt}\PY{p}{(}\PY{l+m+mi}{2}\PY{p}{)}\PY{o}{*}\PY{n}{sqrt}\PY{p}{(}\PY{l+m+mi}{3}\PY{p}{)} \PY{o}{\PYZhy{}} \PY{n}{sqrt}\PY{p}{(}\PY{l+m+mi}{6}\PY{p}{)}\PY{p}{)}
         \PY{n}{display}\PY{p}{(}\PY{n}{sqrt}\PY{p}{(}\PY{l+m+mi}{2}\PY{p}{)}\PY{o}{*}\PY{n}{sqrt}\PY{p}{(}\PY{l+m+mi}{3}\PY{p}{)} \PY{o}{==} \PY{n}{sqrt}\PY{p}{(}\PY{l+m+mi}{6}\PY{p}{)}\PY{p}{)}
\end{Verbatim}


    $$0$$

    
    
    \begin{verbatim}
True
    \end{verbatim}

    
    Calcium will attempt to eliminate more complex extension numbers when it
encounters extension numbers that are algebraically dependent. Here, it
eliminates \(\sqrt{6}\) from the expression, writing the result in terms
of \(\sqrt{2}\) and \(\sqrt{3}\)

    \begin{Verbatim}[commandchars=\\\{\}]
{\color{incolor}In [{\color{incolor}27}]:} \PY{n}{sqrt}\PY{p}{(}\PY{l+m+mi}{2}\PY{p}{)}\PY{o}{*}\PY{n}{sqrt}\PY{p}{(}\PY{l+m+mi}{3}\PY{p}{)} \PY{o}{+} \PY{n}{sqrt}\PY{p}{(}\PY{l+m+mi}{6}\PY{p}{)}
\end{Verbatim}

\texttt{\color{outcolor}Out[{\color{outcolor}27}]:}
    
    $$2 a_{2} a_{3}\; \text{ where } a_{1} = \sqrt{6},\;a_{2} = \sqrt{3},\;a_{3} = \sqrt{2}$$

    

    (Implementation detail: the output shows that \(\sqrt{6}\) is still kept
as part of the field structure, to aid future simplifications, although
it is unused in this particular field element.)

In many cases, it would be desirable to write the above result as an
element of the simple algebraic number field \(\mathbb{Q}(\sqrt{6})\) or
\(\mathbb{Q}(\sqrt{2} + \sqrt{3})\) instead. Calcium will always stay
within a univariate field when performing field operations starting from
a single extension number, but it will not automatically reduce
multivariate number fields to univariate fields. In the future, Calcium
will offer more customizability so that the user can choose between
different behaviors concerning simplification of extension numbers and
fields.

    \subsubsection{Predicates; contexts
objects}\label{predicates-contexts-objects}

    Predicates in Calcium have mathematically semantics: they return True or
False only if Calcium can prove the result. When the truth value is
unknown, Calcium says so explicitly; in \texttt{pyca}, this is done by
raising an exception (\texttt{NotImplementedError}). Consider, as an
example, testing whether \(\exp(\varepsilon) = 1\) where \(\varepsilon\)
is a small number. With \(\varepsilon = 10^{-1000}\), Calcium finds that
the numbers are not equal:

    \begin{Verbatim}[commandchars=\\\{\}]
{\color{incolor}In [{\color{incolor}28}]:} \PY{n}{eps} \PY{o}{=} \PY{n}{ca}\PY{p}{(}\PY{l+m+mi}{10}\PY{p}{)} \PY{o}{*}\PY{o}{*} \PY{p}{(}\PY{o}{\PYZhy{}}\PY{l+m+mi}{1000}\PY{p}{)}
         \PY{n}{exp}\PY{p}{(}\PY{n}{eps}\PY{p}{)} \PY{o}{==} \PY{l+m+mi}{1}
\end{Verbatim}


\begin{Verbatim}[commandchars=\\\{\}]
{\color{outcolor}Out[{\color{outcolor}28}]:} False
\end{Verbatim}
            
    With \(\varepsilon = 10^{-10000}\), Calcium fails:

    \begin{Verbatim}[commandchars=\\\{\}]
{\color{incolor}In [{\color{incolor}29}]:} \PY{n}{eps} \PY{o}{=} \PY{n}{ca}\PY{p}{(}\PY{l+m+mi}{10}\PY{p}{)} \PY{o}{*}\PY{o}{*} \PY{p}{(}\PY{o}{\PYZhy{}}\PY{l+m+mi}{10000}\PY{p}{)}
         \PY{k}{try}\PY{p}{:}
             \PY{n}{exp}\PY{p}{(}\PY{n}{eps}\PY{p}{)} \PY{o}{==} \PY{l+m+mi}{1}
         \PY{k}{except} \PY{n+ne}{Exception} \PY{k}{as} \PY{n}{e}\PY{p}{:}
             \PY{n+nb}{print}\PY{p}{(}\PY{n}{e}\PY{p}{)}
\end{Verbatim}


    \begin{Verbatim}[commandchars=\\\{\}]
unable to decide predicate: equal

    \end{Verbatim}

    The comparison fails because the internal precision limit for numerical
evaluation has been exceeded. The precision limit and many other
settings are stored in a context object, which also serves as a cache of
computed data (such as extension numbers and extension fields). The
context object has type \texttt{ca\_ctx}. There is a default context
called \texttt{ctx\_default}, with the following settings:

    \begin{Verbatim}[commandchars=\\\{\}]
{\color{incolor}In [{\color{incolor}30}]:} \PY{n}{ctx\PYZus{}default}
\end{Verbatim}


\begin{Verbatim}[commandchars=\\\{\}]
{\color{outcolor}Out[{\color{outcolor}30}]:} ca\_ctx(verbose=0, print\_flags=3, mpoly\_ord=0, prec\_limit=4096, qqbar\_deg\_limit=120, low\_prec=64, smooth\_limit=32, lll\_prec=128, pow\_limit=20, use\_gb=1, gb\_length\_limit=100, gb\_poly\_length\_limit=1000, gb\_poly\_bits\_limit=10000, vieta\_limit=6)
\end{Verbatim}
            
    If we create a new context object with higher \texttt{prec\_limit} than
the default value of 4096 bits, the computation succeeds:

    \begin{Verbatim}[commandchars=\\\{\}]
{\color{incolor}In [{\color{incolor}31}]:} \PY{n}{ctx} \PY{o}{=} \PY{n}{ca\PYZus{}ctx}\PY{p}{(}\PY{n}{prec\PYZus{}limit}\PY{o}{=}\PY{l+m+mi}{65536}\PY{p}{)}
         \PY{n}{eps} \PY{o}{=} \PY{n}{ca}\PY{p}{(}\PY{l+m+mi}{10}\PY{p}{,} \PY{n}{context}\PY{o}{=}\PY{n}{ctx}\PY{p}{)} \PY{o}{*}\PY{o}{*} \PY{p}{(}\PY{o}{\PYZhy{}}\PY{l+m+mi}{10000}\PY{p}{)}
         \PY{n}{exp}\PY{p}{(}\PY{n}{eps}\PY{p}{)} \PY{o}{==} \PY{l+m+mi}{1}
\end{Verbatim}


\begin{Verbatim}[commandchars=\\\{\}]
{\color{outcolor}Out[{\color{outcolor}31}]:} False
\end{Verbatim}
            
    The intention is that the user will be able to create multiple "Calcium
fields" for different purposes. Right now, context objects support only
limited configurability.

    \subsection{Algebraic number
identities}\label{algebraic-number-identities}

    Calcium implements a complete decision procedure for testing equality
(or inequality) of algebraic numbers. This functionality is accessible
using either \texttt{qqbar} or \texttt{ca} operations.

Rob Corless proposed checking the following identity from Bill Gosper.
We will first construct the LHS and RHS as symbolic expressions. We
could evaluate them directly with \texttt{qqbar} or \texttt{ca}
operations, which would be more efficient, but this way we can print the
input.

    \begin{Verbatim}[commandchars=\\\{\}]
{\color{incolor}In [{\color{incolor}32}]:} \PY{n}{I} \PY{o}{=} \PY{n}{NumberI}
         
         \PY{n}{lhs} \PY{o}{=} \PY{n}{Sqrt}\PY{p}{(}\PY{l+m+mi}{36} \PY{o}{+} \PY{l+m+mi}{3}\PY{o}{*}\PY{p}{(}\PY{o}{\PYZhy{}}\PY{l+m+mi}{54}\PY{o}{+}\PY{l+m+mi}{35}\PY{o}{*}\PY{n}{I}\PY{o}{*}\PY{n}{Sqrt}\PY{p}{(}\PY{l+m+mi}{3}\PY{p}{)}\PY{p}{)}\PY{o}{*}\PY{o}{*}\PY{n}{Div}\PY{p}{(}\PY{l+m+mi}{1}\PY{p}{,}\PY{l+m+mi}{3}\PY{p}{)}\PY{o}{*}\PY{l+m+mi}{3}\PY{o}{*}\PY{o}{*}\PY{n}{Div}\PY{p}{(}\PY{l+m+mi}{1}\PY{p}{,}\PY{l+m+mi}{3}\PY{p}{)} \PY{o}{+} \PYZbs{}
                     \PY{l+m+mi}{117}\PY{o}{/}\PY{p}{(}\PY{o}{\PYZhy{}}\PY{l+m+mi}{162}\PY{o}{+}\PY{l+m+mi}{105}\PY{o}{*}\PY{n}{I}\PY{o}{*}\PY{n}{Sqrt}\PY{p}{(}\PY{l+m+mi}{3}\PY{p}{)}\PY{p}{)}\PY{o}{*}\PY{o}{*}\PY{n}{Div}\PY{p}{(}\PY{l+m+mi}{1}\PY{p}{,}\PY{l+m+mi}{3}\PY{p}{)}\PY{p}{)}\PY{o}{/}\PY{l+m+mi}{3} \PY{o}{+} \PYZbs{}
                     \PY{n}{Sqrt}\PY{p}{(}\PY{l+m+mi}{5}\PY{p}{)}\PY{o}{*}\PY{p}{(}\PY{l+m+mi}{1296}\PY{o}{*}\PY{n}{I}\PY{o}{+}\PY{l+m+mi}{840}\PY{o}{*}\PY{n}{Sqrt}\PY{p}{(}\PY{l+m+mi}{3}\PY{p}{)}\PY{o}{\PYZhy{}}\PY{l+m+mi}{35}\PY{o}{*}\PY{l+m+mi}{3}\PY{o}{*}\PY{o}{*}\PY{n}{Div}\PY{p}{(}\PY{l+m+mi}{5}\PY{p}{,}\PY{l+m+mi}{6}\PY{p}{)}\PY{o}{*}\PY{p}{(}\PY{o}{\PYZhy{}}\PY{l+m+mi}{54}\PY{o}{+}\PY{l+m+mi}{35}\PY{o}{*}\PY{n}{I}\PY{o}{*}\PY{n}{Sqrt}\PY{p}{(}\PY{l+m+mi}{3}\PY{p}{)}\PY{p}{)}\PY{o}{*}\PY{o}{*}\PY{n}{Div}\PY{p}{(}\PY{l+m+mi}{1}\PY{p}{,}\PY{l+m+mi}{3}\PY{p}{)}\PY{o}{\PYZhy{}}\PYZbs{}
                     \PY{l+m+mi}{54}\PY{o}{*}\PY{n}{I}\PY{o}{*}\PY{p}{(}\PY{o}{\PYZhy{}}\PY{l+m+mi}{162}\PY{o}{+}\PY{l+m+mi}{105}\PY{o}{*}\PY{n}{I}\PY{o}{*}\PY{n}{Sqrt}\PY{p}{(}\PY{l+m+mi}{3}\PY{p}{)}\PY{p}{)}\PY{o}{*}\PY{o}{*}\PY{n}{Div}\PY{p}{(}\PY{l+m+mi}{1}\PY{p}{,}\PY{l+m+mi}{3}\PY{p}{)}\PY{o}{+}\PY{l+m+mi}{13}\PY{o}{*}\PY{n}{I}\PY{o}{*}\PY{p}{(}\PY{o}{\PYZhy{}}\PY{l+m+mi}{162}\PY{o}{+}\PY{l+m+mi}{105}\PY{o}{*}\PY{n}{I}\PY{o}{*}\PY{n}{Sqrt}\PY{p}{(}\PY{l+m+mi}{3}\PY{p}{)}\PY{p}{)}\PY{o}{*}\PY{o}{*}\PY{n}{Div}\PY{p}{(}\PY{l+m+mi}{2}\PY{p}{,}\PY{l+m+mi}{3}\PY{p}{)}\PY{p}{)}\PY{o}{/}\PY{p}{(}\PY{l+m+mi}{5}\PY{o}{*}\PY{p}{(}\PY{l+m+mi}{162}\PY{o}{*}\PY{n}{I}\PY{o}{+}\PY{l+m+mi}{105}\PY{o}{*}\PY{n}{Sqrt}\PY{p}{(}\PY{l+m+mi}{3}\PY{p}{)}\PY{p}{)}\PY{p}{)}
         
         \PY{n}{rhs} \PY{o}{=} \PY{n}{Sqrt}\PY{p}{(}\PY{l+m+mi}{5}\PY{p}{)} \PY{o}{+} \PY{n}{Sqrt}\PY{p}{(}\PY{l+m+mi}{7}\PY{p}{)}
         
         \PY{n}{display}\PY{p}{(}\PY{n}{Equal}\PY{p}{(}\PY{n}{lhs}\PY{p}{,} \PY{n}{rhs}\PY{p}{)}\PY{p}{)}
\end{Verbatim}


    $$\frac{\sqrt{36 + 3 {\left(-54 + 35 i \sqrt{3}\right)}^{1 / 3} \cdot {3}^{1 / 3} + \frac{117}{{\left(-162 + 105 i \sqrt{3}\right)}^{1 / 3}}}}{3} + \frac{\sqrt{5} \left(1296 i + 840 \sqrt{3} - 35 \cdot {3}^{5 / 6} {\left(-54 + 35 i \sqrt{3}\right)}^{1 / 3} - 54 i {\left(-162 + 105 i \sqrt{3}\right)}^{1 / 3} + 13 i {\left(-162 + 105 i \sqrt{3}\right)}^{2 / 3}\right)}{5 \left(162 i + 105 \sqrt{3}\right)} = \sqrt{5} + \sqrt{7}$$

    
    We can check numerically that the expressions agree:

    \begin{Verbatim}[commandchars=\\\{\}]
{\color{incolor}In [{\color{incolor}33}]:} \PY{n+nb}{print}\PY{p}{(}\PY{n}{lhs}\PY{o}{.}\PY{n}{nstr}\PY{p}{(}\PY{p}{)}\PY{p}{)}\PY{p}{;} \PY{n+nb}{print}\PY{p}{(}\PY{n}{rhs}\PY{o}{.}\PY{n}{nstr}\PY{p}{(}\PY{p}{)}\PY{p}{)}
\end{Verbatim}


    \begin{Verbatim}[commandchars=\\\{\}]
4.881819288564380 - 0e-20*I
4.881819288564380

    \end{Verbatim}

    Evaluating the expressions as \texttt{qqbar}s gives the same result,
proving the identity:

    \begin{Verbatim}[commandchars=\\\{\}]
{\color{incolor}In [{\color{incolor}34}]:} \PY{n}{display}\PY{p}{(}\PY{n}{qqbar}\PY{p}{(}\PY{n}{lhs}\PY{p}{)}\PY{p}{)}\PY{p}{;} \PY{n}{display}\PY{p}{(}\PY{n}{qqbar}\PY{p}{(}\PY{n}{rhs}\PY{p}{)}\PY{p}{)}
\end{Verbatim}


    $$\left(\text{Root }\, x \approx {4.88182} \;\text{ of } \;{ x^{4}-24 x^{2}+4}\right)$$

    
    $$\left(\text{Root }\, x \approx {4.88182} \;\text{ of } \;{ x^{4}-24 x^{2}+4}\right)$$

    
    Explicitly:

    \begin{Verbatim}[commandchars=\\\{\}]
{\color{incolor}In [{\color{incolor}35}]:} \PY{n}{qqbar}\PY{p}{(}\PY{n}{lhs}\PY{p}{)} \PY{o}{==} \PY{n}{qqbar}\PY{p}{(}\PY{n}{rhs}\PY{p}{)}
\end{Verbatim}


\begin{Verbatim}[commandchars=\\\{\}]
{\color{outcolor}Out[{\color{outcolor}35}]:} True
\end{Verbatim}
            
    We can also perform the verification using \texttt{ca} arithmetic. The
equality is not immediately apparent since the two expressions evaluate
to elements of different extension fields of \(\mathbb{Q}\):

    \begin{Verbatim}[commandchars=\\\{\}]
{\color{incolor}In [{\color{incolor}36}]:} \PY{n}{display}\PY{p}{(}\PY{n}{ca}\PY{p}{(}\PY{n}{lhs}\PY{p}{)}\PY{p}{)}
\end{Verbatim}


    $$\frac{5 a_{1}- a^{2}_{4} a^{2}_{6} a_{7} + 24 a_{4} a_{6} a_{7}-39 a_{7}}{15 a_{4} a_{6}}\; \text{ where } a_{1} = \sqrt{9 a^{3}_{4} + 36 a^{2}_{4} a^{2}_{6} + 117 a_{4} a_{6}},\;a_{2} = {\left(105 a_{8} a_{9}-162\right)}^{2 / 3},\;a_{3} = {\left(105 a_{8} a_{9}-162\right)}^{1 / 3},\;a_{4} = {\left(35 a_{8} a_{9}-54\right)}^{1 / 3},\;a_{5} = {3}^{5 / 6},\;a_{6} = {3}^{1 / 3},\;a_{7} = \sqrt{5},\;a_{8} = \sqrt{3},\;a_{9} = i$$

    
    \begin{Verbatim}[commandchars=\\\{\}]
{\color{incolor}In [{\color{incolor}37}]:} \PY{n}{display}\PY{p}{(}\PY{n}{ca}\PY{p}{(}\PY{n}{rhs}\PY{p}{)}\PY{p}{)}
\end{Verbatim}


    $$a_{1} + a_{2}\; \text{ where } a_{1} = \sqrt{7},\;a_{2} = \sqrt{5}$$

    
    Fortunately, the equality operator manages to prove that the values are
really equal:

    \begin{Verbatim}[commandchars=\\\{\}]
{\color{incolor}In [{\color{incolor}38}]:} \PY{n}{ca}\PY{p}{(}\PY{n}{lhs}\PY{p}{)} \PY{o}{==} \PY{n}{ca}\PY{p}{(}\PY{n}{rhs}\PY{p}{)}
\end{Verbatim}


\begin{Verbatim}[commandchars=\\\{\}]
{\color{outcolor}Out[{\color{outcolor}38}]:} True
\end{Verbatim}
            
    Indeed, in this case \texttt{ca} arithmetic simplifies the difference of
the expressions to 0 automatically:

    \begin{Verbatim}[commandchars=\\\{\}]
{\color{incolor}In [{\color{incolor}39}]:} \PY{n}{ca}\PY{p}{(}\PY{n}{lhs}\PY{p}{)} \PY{o}{\PYZhy{}} \PY{n}{ca}\PY{p}{(}\PY{n}{rhs}\PY{p}{)}
\end{Verbatim}

\texttt{\color{outcolor}Out[{\color{outcolor}39}]:}
    
    $$0$$

    

    If this example did not look challenging enough, you may try proving
equality of the two huge expressions (involving 7000 operations) given
in
https://ask.sagemath.org/question/52653/equality-of-algebraic-numbers-given-by-huge-symbolic-expressions/.
You will need a bit of string preprocessing to convert the given Sage
expressions to \texttt{fexpr} expressions or ordinary Python syntax for
use with the \texttt{ca} or \texttt{qqbar} types.

This test problem is implemented in the \texttt{huge\_expr.c} program in
the Calcium examples directory. It should take a few seconds to run.

    \subsection{Transcendental number
identities}\label{transcendental-number-identities}

    Calcium is capable of manipulating numbers involving some transcendental
functions such as \texttt{exp} and \texttt{log}. Contrary to the case of
algebraic numbers, it does not have a complete decision procedure for
transcendental numbers, but it has some decent heuristics.

    \subsubsection{Exp and log numbers}\label{exp-and-log-numbers}

Basic logarithmic and exponential simplifications work as expected:

    \begin{Verbatim}[commandchars=\\\{\}]
{\color{incolor}In [{\color{incolor}40}]:} \PY{n}{log}\PY{p}{(}\PY{l+m+mi}{10}\PY{o}{*}\PY{o}{*}\PY{l+m+mi}{20}\PY{p}{)} \PY{o}{/} \PY{n}{log}\PY{p}{(}\PY{l+m+mi}{100}\PY{p}{)}
\end{Verbatim}

\texttt{\color{outcolor}Out[{\color{outcolor}40}]:}
    
    $$10$$

    

    \begin{Verbatim}[commandchars=\\\{\}]
{\color{incolor}In [{\color{incolor}41}]:} \PY{n}{exp}\PY{p}{(}\PY{n}{pi}\PY{p}{)} \PY{o}{*} \PY{n}{exp}\PY{p}{(}\PY{o}{\PYZhy{}}\PY{n}{pi} \PY{o}{+} \PY{n}{log}\PY{p}{(}\PY{l+m+mi}{2}\PY{p}{)}\PY{p}{)}
\end{Verbatim}

\texttt{\color{outcolor}Out[{\color{outcolor}41}]:}
    
    $$2$$

    

    Different formulas for the same number may result in different internal
field representations, in which case Calcium may yet be able to
recognize relations when the numbers are subtracted or divided or when
evaluating a predicate:

    \begin{Verbatim}[commandchars=\\\{\}]
{\color{incolor}In [{\color{incolor}42}]:} \PY{n}{display}\PY{p}{(}\PY{n}{log}\PY{p}{(}\PY{n}{sqrt}\PY{p}{(}\PY{l+m+mi}{2}\PY{p}{)}\PY{p}{)}\PY{p}{)}
         \PY{n}{display}\PY{p}{(}\PY{n}{log}\PY{p}{(}\PY{l+m+mi}{2}\PY{p}{)}\PY{o}{/}\PY{l+m+mi}{2}\PY{p}{)}
         \PY{n}{display}\PY{p}{(}\PY{n}{log}\PY{p}{(}\PY{n}{sqrt}\PY{p}{(}\PY{l+m+mi}{2}\PY{p}{)}\PY{p}{)} \PY{o}{\PYZhy{}} \PY{n}{log}\PY{p}{(}\PY{l+m+mi}{2}\PY{p}{)}\PY{o}{/}\PY{l+m+mi}{2}\PY{p}{)}
         \PY{n}{display}\PY{p}{(}\PY{n}{log}\PY{p}{(}\PY{n}{sqrt}\PY{p}{(}\PY{l+m+mi}{2}\PY{p}{)}\PY{p}{)} \PY{o}{/} \PY{p}{(}\PY{n}{log}\PY{p}{(}\PY{l+m+mi}{2}\PY{p}{)}\PY{o}{/}\PY{l+m+mi}{2}\PY{p}{)}\PY{p}{)}
         \PY{n}{display}\PY{p}{(}\PY{n}{log}\PY{p}{(}\PY{n}{sqrt}\PY{p}{(}\PY{l+m+mi}{2}\PY{p}{)}\PY{p}{)} \PY{o}{*} \PY{p}{(}\PY{n}{log}\PY{p}{(}\PY{l+m+mi}{2}\PY{p}{)}\PY{o}{/}\PY{l+m+mi}{2}\PY{p}{)}\PY{p}{)}
         \PY{n}{log}\PY{p}{(}\PY{n}{sqrt}\PY{p}{(}\PY{l+m+mi}{2}\PY{p}{)}\PY{p}{)} \PY{o}{==} \PY{n}{log}\PY{p}{(}\PY{l+m+mi}{2}\PY{p}{)}\PY{o}{/}\PY{l+m+mi}{2}
\end{Verbatim}


    $$a_{1}\; \text{ where } a_{1} = \log(a_{2}),\;a_{2} = \sqrt{2}$$

    
    $$\frac{a_{1}}{2}\; \text{ where } a_{1} = \log(2)$$

    
    $$0$$

    
    $$1$$

    
    $$\frac{a^{2}_{2}}{4}\; \text{ where } a_{1} = \log(a_{3}),\;a_{2} = \log(2),\;a_{3} = \sqrt{2}$$

    
\begin{Verbatim}[commandchars=\\\{\}]
{\color{outcolor}Out[{\color{outcolor}42}]:} True
\end{Verbatim}
            
    Calcium is aware of branch cuts:

    \begin{Verbatim}[commandchars=\\\{\}]
{\color{incolor}In [{\color{incolor}43}]:} \PY{c+c1}{\PYZsh{} exp(log(x)) == x}
         \PY{n}{display}\PY{p}{(}\PY{n}{exp}\PY{p}{(}\PY{n}{log}\PY{p}{(}\PY{l+m+mi}{1} \PY{o}{+} \PY{l+m+mi}{10}\PY{o}{*}\PY{n}{i}\PY{p}{)}\PY{p}{)}\PY{p}{)}
         \PY{c+c1}{\PYZsh{} log(exp(x)) != x in general; here we get the correct branch!}
         \PY{n}{display}\PY{p}{(}\PY{n}{log}\PY{p}{(}\PY{n}{exp}\PY{p}{(}\PY{l+m+mi}{1} \PY{o}{+} \PY{l+m+mi}{10}\PY{o}{*}\PY{n}{i}\PY{p}{)}\PY{p}{)}\PY{p}{)}
\end{Verbatim}


    $$10 a_{1} + 1\; \text{ where } a_{1} = i$$

    
    $$-4 a_{1} a_{2} + 10 a_{2} + 1\; \text{ where } a_{1} = \pi,\;a_{2} = i$$

    
    Let us check the formulas given on the Calcium documentation front page:

    \begin{Verbatim}[commandchars=\\\{\}]
{\color{incolor}In [{\color{incolor}44}]:} \PY{n}{log}\PY{p}{(}\PY{n}{sqrt}\PY{p}{(}\PY{l+m+mi}{2}\PY{p}{)}\PY{o}{+}\PY{n}{sqrt}\PY{p}{(}\PY{l+m+mi}{3}\PY{p}{)}\PY{p}{)} \PY{o}{/} \PY{n}{log}\PY{p}{(}\PY{l+m+mi}{5} \PY{o}{+} \PY{l+m+mi}{2}\PY{o}{*}\PY{n}{sqrt}\PY{p}{(}\PY{l+m+mi}{6}\PY{p}{)}\PY{p}{)}
\end{Verbatim}

\texttt{\color{outcolor}Out[{\color{outcolor}44}]:}
    
    $$\frac{1}{2}$$

    

    \begin{Verbatim}[commandchars=\\\{\}]
{\color{incolor}In [{\color{incolor}45}]:} \PY{n}{i}\PY{o}{*}\PY{o}{*}\PY{n}{i} \PY{o}{\PYZhy{}} \PY{n}{exp}\PY{p}{(}\PY{n}{pi} \PY{o}{/} \PY{p}{(}\PY{p}{(}\PY{n}{sqrt}\PY{p}{(}\PY{o}{\PYZhy{}}\PY{l+m+mi}{2}\PY{p}{)}\PY{o}{*}\PY{o}{*}\PY{n}{sqrt}\PY{p}{(}\PY{l+m+mi}{2}\PY{p}{)}\PY{p}{)} \PY{o}{*}\PY{o}{*} \PY{n}{sqrt}\PY{p}{(}\PY{l+m+mi}{2}\PY{p}{)}\PY{p}{)}\PY{p}{)}
\end{Verbatim}

\texttt{\color{outcolor}Out[{\color{outcolor}45}]:}
    
    $$0$$

    

    \begin{Verbatim}[commandchars=\\\{\}]
{\color{incolor}In [{\color{incolor}46}]:} \PY{n}{ca}\PY{p}{(}\PY{l+m+mi}{10}\PY{p}{)}\PY{o}{*}\PY{o}{*}\PY{o}{\PYZhy{}}\PY{l+m+mi}{30} \PY{o}{\PYZlt{}} \PY{p}{(}\PY{l+m+mi}{640320}\PY{o}{*}\PY{o}{*}\PY{l+m+mi}{3} \PY{o}{+} \PY{l+m+mi}{744}\PY{p}{)}\PY{o}{/}\PY{n}{exp}\PY{p}{(}\PY{n}{pi}\PY{o}{*}\PY{n}{sqrt}\PY{p}{(}\PY{l+m+mi}{163}\PY{p}{)}\PY{p}{)} \PY{o}{\PYZhy{}} \PY{l+m+mi}{1} \PY{o}{\PYZlt{}} \PY{n}{ca}\PY{p}{(}\PY{l+m+mi}{10}\PY{p}{)}\PY{o}{*}\PY{o}{*}\PY{o}{\PYZhy{}}\PY{l+m+mi}{29}
\end{Verbatim}


\begin{Verbatim}[commandchars=\\\{\}]
{\color{outcolor}Out[{\color{outcolor}46}]:} True
\end{Verbatim}
            
    \subsubsection{Trigonometric functions}\label{trigonometric-functions}

Calcium does not yet have trigonometric or inverse trigonometric
funtions builtin at the C level, but we can "fake" these functions using
complex exponentials and logarithms. The functions \texttt{pyca.sin},
\texttt{pyca.cos}, \texttt{pyca.tan} and \texttt{pyca.atan} do precisely
this.

    \begin{Verbatim}[commandchars=\\\{\}]
{\color{incolor}In [{\color{incolor}47}]:} \PY{n}{sin}\PY{p}{(}\PY{l+m+mi}{3}\PY{p}{)}
\end{Verbatim}

\texttt{\color{outcolor}Out[{\color{outcolor}47}]:}
    
    $$\frac{- a^{2}_{1} a_{2} + a_{2}}{2 a_{1}}\; \text{ where } a_{1} = e^{3 a_{2}},\;a_{2} = i$$

    

    Calcium is capable of proving some trigonometric identities:

    \begin{Verbatim}[commandchars=\\\{\}]
{\color{incolor}In [{\color{incolor}48}]:} \PY{n}{sin}\PY{p}{(}\PY{n}{sqrt}\PY{p}{(}\PY{l+m+mi}{2}\PY{p}{)}\PY{o}{/}\PY{l+m+mi}{2}\PY{p}{)}\PY{o}{*}\PY{o}{*}\PY{l+m+mi}{2} \PY{o}{+} \PY{n}{cos}\PY{p}{(}\PY{l+m+mi}{1}\PY{o}{/}\PY{n}{sqrt}\PY{p}{(}\PY{l+m+mi}{2}\PY{p}{)}\PY{p}{)}\PY{o}{*}\PY{o}{*}\PY{l+m+mi}{2}
\end{Verbatim}

\texttt{\color{outcolor}Out[{\color{outcolor}48}]:}
    
    $$1$$

    

    \begin{Verbatim}[commandchars=\\\{\}]
{\color{incolor}In [{\color{incolor}49}]:} \PY{n}{sin}\PY{p}{(}\PY{l+m+mi}{3} \PY{o}{+} \PY{n}{pi}\PY{p}{)} \PY{o}{+} \PY{n}{sin}\PY{p}{(}\PY{l+m+mi}{3}\PY{p}{)}
\end{Verbatim}

\texttt{\color{outcolor}Out[{\color{outcolor}49}]:}
    
    $$0$$

    

    \begin{Verbatim}[commandchars=\\\{\}]
{\color{incolor}In [{\color{incolor}50}]:} \PY{n}{tan}\PY{p}{(}\PY{n}{atan}\PY{p}{(}\PY{l+m+mi}{5}\PY{p}{)}\PY{p}{)} \PY{o}{==} \PY{l+m+mi}{5}
\end{Verbatim}


\begin{Verbatim}[commandchars=\\\{\}]
{\color{outcolor}Out[{\color{outcolor}50}]:} True
\end{Verbatim}
            
    \begin{Verbatim}[commandchars=\\\{\}]
{\color{incolor}In [{\color{incolor}51}]:} \PY{k}{try}\PY{p}{:}
             \PY{n}{atan}\PY{p}{(}\PY{n}{tan}\PY{p}{(}\PY{l+m+mi}{1}\PY{p}{)}\PY{p}{)} \PY{o}{==} \PY{l+m+mi}{1}
         \PY{k}{except} \PY{n+ne}{NotImplementedError}\PY{p}{:}
             \PY{n+nb}{print}\PY{p}{(}\PY{l+s+s2}{\PYZdq{}}\PY{l+s+s2}{This simplification does not work yet}\PY{l+s+s2}{\PYZdq{}}\PY{p}{)}
\end{Verbatim}


    \begin{Verbatim}[commandchars=\\\{\}]
This simplification does not work yet

    \end{Verbatim}

    We test some simplifications involving the Gudermannian function
\(\operatorname{gd}(x) = 2 \operatorname{atan}(e^x) - \pi/2\):

    \begin{Verbatim}[commandchars=\\\{\}]
{\color{incolor}In [{\color{incolor}52}]:} \PY{k}{def} \PY{n+nf}{gd}\PY{p}{(}\PY{n}{x}\PY{p}{)}\PY{p}{:}
             \PY{k}{return} \PY{l+m+mi}{2}\PY{o}{*}\PY{n}{atan}\PY{p}{(}\PY{n}{exp}\PY{p}{(}\PY{n}{x}\PY{p}{)}\PY{p}{)}\PY{o}{\PYZhy{}}\PY{n}{pi}\PY{o}{/}\PY{l+m+mi}{2}
         
         \PY{n}{display}\PY{p}{(}\PY{n}{sin}\PY{p}{(}\PY{n}{gd}\PY{p}{(}\PY{l+m+mi}{1}\PY{p}{)}\PY{p}{)} \PY{o}{\PYZhy{}} \PY{n}{tanh}\PY{p}{(}\PY{l+m+mi}{1}\PY{p}{)}\PY{p}{)}
         \PY{n}{display}\PY{p}{(}\PY{n}{tan}\PY{p}{(}\PY{n}{gd}\PY{p}{(}\PY{l+m+mi}{1}\PY{p}{)}\PY{p}{)} \PY{o}{\PYZhy{}} \PY{n}{sinh}\PY{p}{(}\PY{l+m+mi}{1}\PY{p}{)}\PY{p}{)}
         \PY{n}{display}\PY{p}{(}\PY{n}{sin}\PY{p}{(}\PY{n}{gd}\PY{p}{(}\PY{n}{sqrt}\PY{p}{(}\PY{l+m+mi}{2}\PY{p}{)}\PY{p}{)}\PY{p}{)} \PY{o}{\PYZhy{}} \PY{n}{tanh}\PY{p}{(}\PY{n}{sqrt}\PY{p}{(}\PY{l+m+mi}{2}\PY{p}{)}\PY{p}{)}\PY{p}{)}
         \PY{n}{display}\PY{p}{(}\PY{n}{tan}\PY{p}{(}\PY{n}{gd}\PY{p}{(}\PY{l+m+mi}{1}\PY{p}{)}\PY{o}{/}\PY{l+m+mi}{2}\PY{p}{)} \PY{o}{\PYZhy{}} \PY{n}{tanh}\PY{p}{(}\PY{n}{ca}\PY{p}{(}\PY{l+m+mi}{1}\PY{p}{)}\PY{o}{/}\PY{l+m+mi}{2}\PY{p}{)}\PY{p}{)}
\end{Verbatim}


    $$0$$

    
    $$0$$

    
    $$0$$

    
    $$0$$

    
    Let us try to prove a famous identity; Machin's formula for \(\pi\):

    \begin{Verbatim}[commandchars=\\\{\}]
{\color{incolor}In [{\color{incolor}53}]:} \PY{n}{lhs} \PY{o}{=} \PY{l+m+mi}{4}\PY{o}{*}\PY{n}{Atan}\PY{p}{(}\PY{n}{Div}\PY{p}{(}\PY{l+m+mi}{1}\PY{p}{,}\PY{l+m+mi}{5}\PY{p}{)}\PY{p}{)} \PY{o}{\PYZhy{}} \PY{n}{Atan}\PY{p}{(}\PY{n}{Div}\PY{p}{(}\PY{l+m+mi}{1}\PY{p}{,}\PY{l+m+mi}{239}\PY{p}{)}\PY{p}{)}
         \PY{n}{rhs} \PY{o}{=} \PY{n}{Pi} \PY{o}{/} \PY{l+m+mi}{4}
         \PY{n}{Equal}\PY{p}{(}\PY{n}{lhs}\PY{p}{,} \PY{n}{rhs}\PY{p}{)}
\end{Verbatim}

\texttt{\color{outcolor}Out[{\color{outcolor}53}]:}
    
    $$4 \operatorname{atan}\!\left(\frac{1}{5}\right) - \operatorname{atan}\!\left(\frac{1}{239}\right) = \frac{\pi}{4}$$

    

    \begin{Verbatim}[commandchars=\\\{\}]
{\color{incolor}In [{\color{incolor}54}]:} \PY{n+nb}{print}\PY{p}{(}\PY{n}{lhs}\PY{o}{.}\PY{n}{nstr}\PY{p}{(}\PY{p}{)}\PY{p}{)}
         \PY{n+nb}{print}\PY{p}{(}\PY{n}{rhs}\PY{o}{.}\PY{n}{nstr}\PY{p}{(}\PY{p}{)}\PY{p}{)}
\end{Verbatim}


    \begin{Verbatim}[commandchars=\\\{\}]
0.7853981633974483
0.7853981633974483

    \end{Verbatim}

    Evaluating the left-hand side using \texttt{ca} arithmetic gives us
nothing as simple as \(\pi / 4\):

    \begin{Verbatim}[commandchars=\\\{\}]
{\color{incolor}In [{\color{incolor}55}]:} \PY{l+m+mi}{4}\PY{o}{*}\PY{n}{atan}\PY{p}{(}\PY{n}{ca}\PY{p}{(}\PY{l+m+mi}{1}\PY{p}{)}\PY{o}{/}\PY{l+m+mi}{5}\PY{p}{)} \PY{o}{\PYZhy{}} \PY{n}{atan}\PY{p}{(}\PY{n}{ca}\PY{p}{(}\PY{l+m+mi}{1}\PY{p}{)}\PY{o}{/}\PY{l+m+mi}{239}\PY{p}{)}
\end{Verbatim}

\texttt{\color{outcolor}Out[{\color{outcolor}55}]:}
    
    $$\frac{a_{1} a_{3}-4 a_{2} a_{3}}{2}\; \text{ where } a_{1} = \log\!\left(\frac{239 a_{3} + 28560}{28561}\right),\;a_{2} = \log\!\left(\frac{5 a_{3} + 12}{13}\right),\;a_{3} = i$$

    

    Nevertheless, Calcium finds the identity when \(\pi\) is given as part
of the input:

    \begin{Verbatim}[commandchars=\\\{\}]
{\color{incolor}In [{\color{incolor}56}]:} \PY{l+m+mi}{4}\PY{o}{*}\PY{n}{atan}\PY{p}{(}\PY{n}{ca}\PY{p}{(}\PY{l+m+mi}{1}\PY{p}{)}\PY{o}{/}\PY{l+m+mi}{5}\PY{p}{)} \PY{o}{\PYZhy{}} \PY{n}{atan}\PY{p}{(}\PY{n}{ca}\PY{p}{(}\PY{l+m+mi}{1}\PY{p}{)}\PY{o}{/}\PY{l+m+mi}{239}\PY{p}{)} \PY{o}{\PYZhy{}} \PY{n}{pi}\PY{o}{/}\PY{l+m+mi}{4}
\end{Verbatim}

\texttt{\color{outcolor}Out[{\color{outcolor}56}]:}
    
    $$0$$

    

    \begin{Verbatim}[commandchars=\\\{\}]
{\color{incolor}In [{\color{incolor}57}]:} \PY{l+m+mi}{4}\PY{o}{*}\PY{n}{atan}\PY{p}{(}\PY{n}{ca}\PY{p}{(}\PY{l+m+mi}{1}\PY{p}{)}\PY{o}{/}\PY{l+m+mi}{5}\PY{p}{)} \PY{o}{\PYZhy{}} \PY{n}{atan}\PY{p}{(}\PY{n}{ca}\PY{p}{(}\PY{l+m+mi}{1}\PY{p}{)}\PY{o}{/}\PY{l+m+mi}{239}\PY{p}{)} \PY{o}{==} \PY{n}{pi}\PY{o}{/}\PY{l+m+mi}{4}
\end{Verbatim}


\begin{Verbatim}[commandchars=\\\{\}]
{\color{outcolor}Out[{\color{outcolor}57}]:} True
\end{Verbatim}
            
    Here is a more complicated formula:

    \begin{Verbatim}[commandchars=\\\{\}]
{\color{incolor}In [{\color{incolor}58}]:} \PY{l+m+mi}{12}\PY{o}{*}\PY{n}{atan}\PY{p}{(}\PY{n}{ca}\PY{p}{(}\PY{l+m+mi}{1}\PY{p}{)}\PY{o}{/}\PY{l+m+mi}{49}\PY{p}{)} \PY{o}{+} \PY{l+m+mi}{32}\PY{o}{*}\PY{n}{atan}\PY{p}{(}\PY{n}{ca}\PY{p}{(}\PY{l+m+mi}{1}\PY{p}{)}\PY{o}{/}\PY{l+m+mi}{57}\PY{p}{)} \PY{o}{\PYZhy{}} \PY{l+m+mi}{5}\PY{o}{*}\PY{n}{atan}\PY{p}{(}\PY{n}{ca}\PY{p}{(}\PY{l+m+mi}{1}\PY{p}{)}\PY{o}{/}\PY{l+m+mi}{239}\PY{p}{)} \PY{o}{+} \PY{l+m+mi}{12}\PY{o}{*}\PY{n}{atan}\PY{p}{(}\PY{n}{ca}\PY{p}{(}\PY{l+m+mi}{1}\PY{p}{)}\PY{o}{/}\PY{l+m+mi}{110443}\PY{p}{)}
\end{Verbatim}

\texttt{\color{outcolor}Out[{\color{outcolor}58}]:}
    
    $$\frac{-12 a_{1} a_{5} + 5 a_{2} a_{5}-32 a_{3} a_{5}-12 a_{4} a_{5}}{2}\; \text{ where } a_{1} = \log\!\left(\frac{110443 a_{5} + 6098828124}{6098828125}\right),\;a_{2} = \log\!\left(\frac{239 a_{5} + 28560}{28561}\right),\;a_{3} = \log\!\left(\frac{57 a_{5} + 1624}{1625}\right),\;a_{4} = \log\!\left(\frac{49 a_{5} + 1200}{1201}\right),\;a_{5} = i$$

    

    \begin{Verbatim}[commandchars=\\\{\}]
{\color{incolor}In [{\color{incolor}59}]:} \PY{l+m+mi}{12}\PY{o}{*}\PY{n}{atan}\PY{p}{(}\PY{n}{ca}\PY{p}{(}\PY{l+m+mi}{1}\PY{p}{)}\PY{o}{/}\PY{l+m+mi}{49}\PY{p}{)} \PY{o}{+} \PY{l+m+mi}{32}\PY{o}{*}\PY{n}{atan}\PY{p}{(}\PY{n}{ca}\PY{p}{(}\PY{l+m+mi}{1}\PY{p}{)}\PY{o}{/}\PY{l+m+mi}{57}\PY{p}{)} \PY{o}{\PYZhy{}} \PY{l+m+mi}{5}\PY{o}{*}\PY{n}{atan}\PY{p}{(}\PY{n}{ca}\PY{p}{(}\PY{l+m+mi}{1}\PY{p}{)}\PY{o}{/}\PY{l+m+mi}{239}\PY{p}{)} \PY{o}{+} \PY{l+m+mi}{12}\PY{o}{*}\PY{n}{atan}\PY{p}{(}\PY{n}{ca}\PY{p}{(}\PY{l+m+mi}{1}\PY{p}{)}\PY{o}{/}\PY{l+m+mi}{110443}\PY{p}{)} \PY{o}{\PYZhy{}} \PY{n}{pi}\PY{o}{/}\PY{l+m+mi}{4}
\end{Verbatim}

\texttt{\color{outcolor}Out[{\color{outcolor}59}]:}
    
    $$0$$

    

    Hyperbolic formulas also work:

    \begin{Verbatim}[commandchars=\\\{\}]
{\color{incolor}In [{\color{incolor}60}]:} \PY{n}{atanh} \PY{o}{=} \PY{k}{lambda} \PY{n}{x}\PY{p}{:} \PY{o}{\PYZhy{}}\PY{n}{i}\PY{o}{*}\PY{n}{atan}\PY{p}{(}\PY{n}{i}\PY{o}{*}\PY{n}{x}\PY{p}{)}
         \PY{l+m+mi}{32}\PY{o}{*}\PY{n}{atanh}\PY{p}{(}\PY{n}{ca}\PY{p}{(}\PY{l+m+mi}{1}\PY{p}{)}\PY{o}{/}\PY{l+m+mi}{31}\PY{p}{)} \PY{o}{+} \PY{l+m+mi}{24}\PY{o}{*}\PY{n}{atanh}\PY{p}{(}\PY{n}{ca}\PY{p}{(}\PY{l+m+mi}{1}\PY{p}{)}\PY{o}{/}\PY{l+m+mi}{49}\PY{p}{)} \PY{o}{+} \PY{l+m+mi}{14}\PY{o}{*}\PY{n}{atanh}\PY{p}{(}\PY{n}{ca}\PY{p}{(}\PY{l+m+mi}{1}\PY{p}{)}\PY{o}{/}\PY{l+m+mi}{161}\PY{p}{)}
\end{Verbatim}

\texttt{\color{outcolor}Out[{\color{outcolor}60}]:}
    
    $$-7 a_{1}-12 a_{2}-16 a_{3}\; \text{ where } a_{1} = \log\!\left(\frac{80}{81}\right),\;a_{2} = \log\!\left(\frac{24}{25}\right),\;a_{3} = \log\!\left(\frac{15}{16}\right),\;a_{4} = i$$

    

    \begin{Verbatim}[commandchars=\\\{\}]
{\color{incolor}In [{\color{incolor}61}]:} \PY{l+m+mi}{32}\PY{o}{*}\PY{n}{atanh}\PY{p}{(}\PY{n}{ca}\PY{p}{(}\PY{l+m+mi}{1}\PY{p}{)}\PY{o}{/}\PY{l+m+mi}{31}\PY{p}{)} \PY{o}{+} \PY{l+m+mi}{24}\PY{o}{*}\PY{n}{atanh}\PY{p}{(}\PY{n}{ca}\PY{p}{(}\PY{l+m+mi}{1}\PY{p}{)}\PY{o}{/}\PY{l+m+mi}{49}\PY{p}{)} \PY{o}{+} \PY{l+m+mi}{14}\PY{o}{*}\PY{n}{atanh}\PY{p}{(}\PY{n}{ca}\PY{p}{(}\PY{l+m+mi}{1}\PY{p}{)}\PY{o}{/}\PY{l+m+mi}{161}\PY{p}{)} \PY{o}{\PYZhy{}} \PY{n}{log}\PY{p}{(}\PY{l+m+mi}{5}\PY{p}{)}
\end{Verbatim}

\texttt{\color{outcolor}Out[{\color{outcolor}61}]:}
    
    $$0$$

    

    \subsection{Matrices}\label{matrices}

The \texttt{ca\_mat} type provides matrices with \texttt{ca} entries. We
look at some examples of basic manipulation:

    \begin{Verbatim}[commandchars=\\\{\}]
{\color{incolor}In [{\color{incolor}62}]:} \PY{n}{A} \PY{o}{=} \PY{n}{ca\PYZus{}mat}\PY{p}{(}\PY{p}{[}\PY{p}{[}\PY{l+m+mi}{1}\PY{p}{,} \PY{n}{i}\PY{o}{*}\PY{n}{pi}\PY{p}{]}\PY{p}{,} \PY{p}{[}\PY{o}{\PYZhy{}}\PY{n}{i}\PY{o}{*}\PY{n}{pi}\PY{p}{,} \PY{l+m+mi}{2}\PY{p}{]}\PY{p}{]}\PY{p}{)}
         \PY{n}{A}
\end{Verbatim}

\texttt{\color{outcolor}Out[{\color{outcolor}62}]:}
    
    $$\displaystyle{\begin{pmatrix}1 & a_{1} a_{2} \\- a_{1} a_{2} & 2\end{pmatrix}}\; \text{ where } a_{1} = \pi,\;a_{2} = i$$

    

    \begin{Verbatim}[commandchars=\\\{\}]
{\color{incolor}In [{\color{incolor}63}]:} \PY{n}{A} \PY{o}{*} \PY{n}{A}
\end{Verbatim}

\texttt{\color{outcolor}Out[{\color{outcolor}63}]:}
    
    $$\displaystyle{\begin{pmatrix}a^{2}_{1} + 1 & 3 a_{1} a_{2} \\-3 a_{1} a_{2} & a^{2}_{1} + 4\end{pmatrix}}\; \text{ where } a_{1} = \pi,\;a_{2} = i$$

    

    \begin{Verbatim}[commandchars=\\\{\}]
{\color{incolor}In [{\color{incolor}64}]:} \PY{n}{display}\PY{p}{(}\PY{n}{A}\PY{o}{.}\PY{n}{det}\PY{p}{(}\PY{p}{)}\PY{p}{)}
         \PY{n}{display}\PY{p}{(}\PY{n}{A}\PY{o}{.}\PY{n}{trace}\PY{p}{(}\PY{p}{)}\PY{p}{)}
         \PY{n}{display}\PY{p}{(}\PY{n}{A}\PY{o}{.}\PY{n}{rank}\PY{p}{(}\PY{p}{)}\PY{p}{)}
\end{Verbatim}


    $$- a^{2}_{1} + 2\; \text{ where } a_{1} = \pi,\;a_{2} = i$$

    
    $$3$$

    
    
    \begin{verbatim}
2
    \end{verbatim}

    
    Solving linear systems:

    \begin{Verbatim}[commandchars=\\\{\}]
{\color{incolor}In [{\color{incolor}65}]:} \PY{n}{B} \PY{o}{=} \PY{n}{ca\PYZus{}mat}\PY{p}{(}\PY{p}{[}\PY{p}{[}\PY{l+m+mi}{1}\PY{p}{]}\PY{p}{,} \PY{p}{[}\PY{l+m+mi}{2}\PY{p}{]}\PY{p}{]}\PY{p}{)}
         \PY{n}{X} \PY{o}{=} \PY{n}{A}\PY{o}{.}\PY{n}{solve}\PY{p}{(}\PY{n}{B}\PY{p}{)}
         \PY{n}{display}\PY{p}{(}\PY{n}{X}\PY{p}{)}
         \PY{n}{display}\PY{p}{(}\PY{n}{A} \PY{o}{*} \PY{n}{X}\PY{p}{)}
\end{Verbatim}


    $$\displaystyle{\begin{pmatrix}\frac{2 a_{1} a_{2}-2}{a^{2}_{1}-2} \\\frac{- a_{1} a_{2}-2}{a^{2}_{1}-2}\end{pmatrix}}\; \text{ where } a_{1} = \pi,\;a_{2} = i$$

    
    $$\displaystyle{\begin{pmatrix}1 \\2\end{pmatrix}}$$

    
    Computing row echelon forms and characteristic polynomials:

    \begin{Verbatim}[commandchars=\\\{\}]
{\color{incolor}In [{\color{incolor}66}]:} \PY{n}{ca\PYZus{}mat}\PY{p}{(}\PY{p}{[}\PY{p}{[}\PY{l+m+mi}{1}\PY{p}{,}\PY{n}{pi}\PY{p}{,}\PY{l+m+mi}{2}\PY{p}{,}\PY{n}{pi}\PY{p}{]}\PY{p}{,}\PY{p}{[}\PY{l+m+mi}{1}\PY{o}{/}\PY{n}{pi}\PY{p}{,}\PY{l+m+mi}{3}\PY{p}{,}\PY{l+m+mi}{1}\PY{o}{/}\PY{p}{(}\PY{n}{pi}\PY{o}{+}\PY{l+m+mi}{1}\PY{p}{)}\PY{p}{,}\PY{l+m+mi}{4}\PY{p}{]}\PY{p}{,}\PY{p}{[}\PY{l+m+mi}{1}\PY{p}{,}\PY{l+m+mi}{1}\PY{p}{,}\PY{l+m+mi}{1}\PY{p}{,}\PY{l+m+mi}{1}\PY{p}{]}\PY{p}{]}\PY{p}{)}\PY{o}{.}\PY{n}{rref}\PY{p}{(}\PY{p}{)}
\end{Verbatim}

\texttt{\color{outcolor}Out[{\color{outcolor}66}]:}
    
    $$\displaystyle{\begin{pmatrix}1 & 0 & 0 & \frac{a^{3}_{1}- a^{2}_{1}-2 a_{1}}{3 a^{2}_{1} + 3 a_{1}-2} \\0 & 1 & 0 & \frac{4 a^{2}_{1} + 4 a_{1}-2}{3 a^{2}_{1} + 3 a_{1}-2} \\0 & 0 & 1 & \frac{- a^{3}_{1} + a_{1}}{3 a^{2}_{1} + 3 a_{1}-2}\end{pmatrix}}\; \text{ where } a_{1} = \pi$$

    

    \begin{Verbatim}[commandchars=\\\{\}]
{\color{incolor}In [{\color{incolor}67}]:} \PY{n}{A} \PY{o}{=} \PY{n}{ca\PYZus{}mat}\PY{p}{(}\PY{p}{[}\PY{p}{[}\PY{l+m+mi}{5}\PY{p}{,} \PY{n}{pi}\PY{p}{]}\PY{p}{,} \PY{p}{[}\PY{l+m+mi}{1}\PY{p}{,} \PY{o}{\PYZhy{}}\PY{l+m+mi}{1}\PY{p}{]}\PY{p}{]}\PY{p}{)}\PY{o}{*}\PY{o}{*}\PY{l+m+mi}{4}
         \PY{n}{display}\PY{p}{(}\PY{n}{A}\PY{p}{)}
         \PY{n}{display}\PY{p}{(}\PY{n}{A}\PY{o}{.}\PY{n}{charpoly}\PY{p}{(}\PY{p}{)}\PY{p}{)}
         \PY{n}{display}\PY{p}{(}\PY{n}{A}\PY{o}{.}\PY{n}{charpoly}\PY{p}{(}\PY{p}{)}\PY{p}{(}\PY{n}{A}\PY{p}{)}\PY{p}{)}
\end{Verbatim}


    $$\displaystyle{\begin{pmatrix}a^{2}_{1} + 66 a_{1} + 625 & 8 a^{2}_{1} + 104 a_{1} \\8 a_{1} + 104 & a^{2}_{1} + 18 a_{1} + 1\end{pmatrix}}\; \text{ where } a_{1} = \pi$$

    
    $$ x^{2}+ \left(-2 a^{2}_{1}-84 a_{1}-626\right) x+ \left(a^{4}_{1} + 20 a^{3}_{1} + 150 a^{2}_{1} + 500 a_{1} + 625\right)\; \text{ where } a_{1} = \pi$$

    
    $$\displaystyle{\begin{pmatrix}0 & 0 \\0 & 0\end{pmatrix}}$$

    
    Calcium correctly recognizes singular matrices, even matrices that are
nontrivially singular.

    \begin{Verbatim}[commandchars=\\\{\}]
{\color{incolor}In [{\color{incolor}68}]:} \PY{n}{A} \PY{o}{=} \PY{n}{ca\PYZus{}mat}\PY{p}{(}\PY{p}{[}\PY{p}{[}\PY{n}{pi}\PY{p}{,} \PY{n}{pi}\PY{o}{*}\PY{o}{*}\PY{l+m+mi}{2}\PY{p}{]}\PY{p}{,} \PY{p}{[}\PY{n}{pi}\PY{o}{*}\PY{o}{*}\PY{l+m+mi}{3}\PY{p}{,} \PY{n}{pi}\PY{o}{*}\PY{o}{*}\PY{l+m+mi}{4}\PY{p}{]}\PY{p}{]}\PY{p}{)}
         \PY{n}{A}
\end{Verbatim}

\texttt{\color{outcolor}Out[{\color{outcolor}68}]:}
    
    $$\displaystyle{\begin{pmatrix}a_{1} & a^{2}_{1} \\a^{3}_{1} & a^{4}_{1}\end{pmatrix}}\; \text{ where } a_{1} = \pi$$

    

    \begin{Verbatim}[commandchars=\\\{\}]
{\color{incolor}In [{\color{incolor}69}]:} \PY{k}{try}\PY{p}{:}
             \PY{n}{A}\PY{o}{.}\PY{n}{solve}\PY{p}{(}\PY{n}{ca\PYZus{}mat}\PY{p}{(}\PY{p}{[}\PY{p}{[}\PY{l+m+mi}{1}\PY{p}{]}\PY{p}{,} \PY{p}{[}\PY{l+m+mi}{2}\PY{p}{]}\PY{p}{]}\PY{p}{)}\PY{p}{)}
         \PY{k}{except} \PY{n+ne}{ZeroDivisionError}\PY{p}{:}
             \PY{n+nb}{print}\PY{p}{(}\PY{l+s+s2}{\PYZdq{}}\PY{l+s+s2}{matrix is singular!}\PY{l+s+s2}{\PYZdq{}}\PY{p}{)}
\end{Verbatim}


    \begin{Verbatim}[commandchars=\\\{\}]
matrix is singular!

    \end{Verbatim}

    \begin{Verbatim}[commandchars=\\\{\}]
{\color{incolor}In [{\color{incolor}70}]:} \PY{n}{A}\PY{o}{.}\PY{n}{rref}\PY{p}{(}\PY{p}{)}
\end{Verbatim}

\texttt{\color{outcolor}Out[{\color{outcolor}70}]:}
    
    $$\displaystyle{\begin{pmatrix}1 & a_{1} \\0 & 0\end{pmatrix}}\; \text{ where } a_{1} = \pi$$

    

    \begin{Verbatim}[commandchars=\\\{\}]
{\color{incolor}In [{\color{incolor}71}]:} \PY{n}{A}\PY{o}{.}\PY{n}{rank}\PY{p}{(}\PY{p}{)}
\end{Verbatim}


\begin{Verbatim}[commandchars=\\\{\}]
{\color{outcolor}Out[{\color{outcolor}71}]:} 1
\end{Verbatim}
            
    Exact matrix operations easily result in large expressions:

    \begin{Verbatim}[commandchars=\\\{\}]
{\color{incolor}In [{\color{incolor}72}]:} \PY{n}{A} \PY{o}{=} \PY{n}{ca\PYZus{}mat}\PY{p}{(}\PY{p}{[}\PY{p}{[}\PY{n}{sqrt}\PY{p}{(}\PY{n}{i}\PY{o}{+}\PY{n}{j}\PY{o}{+}\PY{l+m+mi}{1}\PY{p}{)} \PY{k}{for} \PY{n}{i} \PY{o+ow}{in} \PY{n+nb}{range}\PY{p}{(}\PY{l+m+mi}{6}\PY{p}{)}\PY{p}{]} \PY{k}{for} \PY{n}{j} \PY{o+ow}{in} \PY{n+nb}{range}\PY{p}{(}\PY{l+m+mi}{6}\PY{p}{)}\PY{p}{]}\PY{p}{)}
         \PY{n}{A}
\end{Verbatim}

\texttt{\color{outcolor}Out[{\color{outcolor}72}]:}
    
    $$\displaystyle{\begin{pmatrix}1 & a_{7} & a_{6} & 2 & a_{5} & a_{4} \\a_{7} & a_{6} & 2 & a_{5} & a_{4} & a_{3} \\a_{6} & 2 & a_{5} & a_{4} & a_{3} & 2 a_{7} \\2 & a_{5} & a_{4} & a_{3} & 2 a_{7} & 3 \\a_{5} & a_{4} & a_{3} & 2 a_{7} & 3 & a_{2} \\a_{4} & a_{3} & 2 a_{7} & 3 & a_{2} & a_{1}\end{pmatrix}}\; \text{ where } a_{1} = \sqrt{11},\;a_{2} = \sqrt{10},\;a_{3} = \sqrt{7},\;a_{4} = \sqrt{6},\;a_{5} = \sqrt{5},\;a_{6} = \sqrt{3},\;a_{7} = \sqrt{2}$$

    

    \begin{Verbatim}[commandchars=\\\{\}]
{\color{incolor}In [{\color{incolor}73}]:} \PY{n}{A}\PY{o}{.}\PY{n}{det}\PY{p}{(}\PY{p}{)}
\end{Verbatim}

\texttt{\color{outcolor}Out[{\color{outcolor}73}]:}
    
    $$-4 a_{1} a_{3} a_{5} a_{6}-20 a_{1} a_{3} a_{5} a_{7}-24 a_{1} a_{3} a_{5}-4 a_{1} a_{3} a_{6} a_{7} + 8 a_{1} a_{3} a_{6} + 136 a_{1} a_{3}-28 a_{1} a_{5} a_{6} a_{7}-116 a_{1} a_{5} a_{6}-88 a_{1} a_{5} a_{7} + 64 a_{1} a_{5} + 112 a_{1} a_{6} a_{7} + 164 a_{1} a_{6}-60 a_{1} a_{7} + 244 a_{1} + 204 a_{3} a_{5} a_{6} a_{7}-96 a_{3} a_{5} a_{6} + 8 a_{3} a_{5} a_{7} + 144 a_{3} a_{5}-152 a_{3} a_{6} a_{7}-240 a_{3} a_{6} + 500 a_{3} a_{7} + 548 a_{3}-216 a_{5} a_{6} a_{7} + 116 a_{5} a_{6} + 628 a_{5} a_{7} + 764 a_{5}-500 a_{6} a_{7} + 24 a_{6}-1440 a_{7}-3868\; \text{ where } a_{1} = \sqrt{11},\;a_{2} = \sqrt{10},\;a_{3} = \sqrt{7},\;a_{4} = \sqrt{6},\;a_{5} = \sqrt{5},\;a_{6} = \sqrt{3},\;a_{7} = \sqrt{2}$$

    

    \subsubsection{Eigenvalues and matrix
functions}\label{eigenvalues-and-matrix-functions}

    Calcium can calculate exact eigenvalues of matrices, with correct
multiplicities. This is accomplished by factoring the characteristic
polynomial. Currently, computing polynomial roots will only work for
polynomials with very simple structure or with rational entries, so do
not expect too much!

    \begin{Verbatim}[commandchars=\\\{\}]
{\color{incolor}In [{\color{incolor}74}]:} \PY{n}{A} \PY{o}{=} \PY{n}{ca\PYZus{}mat}\PY{p}{(}\PY{p}{[}\PY{p}{[}\PY{l+m+mi}{1}\PY{p}{,}\PY{n}{pi}\PY{p}{]}\PY{p}{,}\PY{p}{[}\PY{o}{\PYZhy{}}\PY{n}{pi}\PY{p}{,}\PY{l+m+mi}{1}\PY{p}{]}\PY{p}{]}\PY{p}{)}
         \PY{n}{display}\PY{p}{(}\PY{n}{A}\PY{p}{)}
         
         \PY{k}{for} \PY{n}{lamda}\PY{p}{,} \PY{n}{mult} \PY{o+ow}{in} \PY{n}{A}\PY{o}{.}\PY{n}{eigenvalues}\PY{p}{(}\PY{p}{)}\PY{p}{:}
             \PY{n+nb}{print}\PY{p}{(}\PY{l+s+s2}{\PYZdq{}}\PY{l+s+s2}{Multiplicity}\PY{l+s+s2}{\PYZdq{}}\PY{p}{,} \PY{n}{mult}\PY{p}{)}
             \PY{n}{display}\PY{p}{(}\PY{n}{lamda}\PY{p}{)}
\end{Verbatim}


    $$\displaystyle{\begin{pmatrix}1 & a_{1} \\- a_{1} & 1\end{pmatrix}}\; \text{ where } a_{1} = \pi$$

    
    \begin{Verbatim}[commandchars=\\\{\}]
Multiplicity 1

    \end{Verbatim}

    $$a_{1} a_{2} + 1\; \text{ where } a_{1} = \pi,\;a_{2} = i$$

    
    \begin{Verbatim}[commandchars=\\\{\}]
Multiplicity 1

    \end{Verbatim}

    $$- a_{1} a_{2} + 1\; \text{ where } a_{1} = \pi,\;a_{2} = i$$

    
    We demonstrate computing the Jordan decomposition of a matrix with
nontrivial Jordan form:

    \begin{Verbatim}[commandchars=\\\{\}]
{\color{incolor}In [{\color{incolor}75}]:} \PY{n}{A} \PY{o}{=} \PY{n}{ca\PYZus{}mat}\PY{p}{(}\PY{p}{[}\PY{p}{[}\PY{l+m+mi}{20}\PY{p}{,}\PY{l+m+mi}{77}\PY{p}{,}\PY{l+m+mi}{59}\PY{p}{,}\PY{l+m+mi}{40}\PY{p}{]}\PY{p}{,} \PY{p}{[}\PY{l+m+mi}{0}\PY{p}{,}\PY{o}{\PYZhy{}}\PY{l+m+mi}{2}\PY{p}{,}\PY{o}{\PYZhy{}}\PY{l+m+mi}{3}\PY{p}{,}\PY{o}{\PYZhy{}}\PY{l+m+mi}{2}\PY{p}{]}\PY{p}{,} \PY{p}{[}\PY{o}{\PYZhy{}}\PY{l+m+mi}{10}\PY{p}{,}\PY{o}{\PYZhy{}}\PY{l+m+mi}{35}\PY{p}{,}\PY{o}{\PYZhy{}}\PY{l+m+mi}{23}\PY{p}{,}\PY{o}{\PYZhy{}}\PY{l+m+mi}{15}\PY{p}{]}\PY{p}{,} \PY{p}{[}\PY{l+m+mi}{2}\PY{p}{,}\PY{l+m+mi}{7}\PY{p}{,}\PY{l+m+mi}{3}\PY{p}{,}\PY{l+m+mi}{1}\PY{p}{]}\PY{p}{]}\PY{p}{)}
         \PY{n}{J}\PY{p}{,} \PY{n}{P} \PY{o}{=} \PY{n}{A}\PY{o}{.}\PY{n}{jordan\PYZus{}form}\PY{p}{(}\PY{n}{transform}\PY{o}{=}\PY{k+kc}{True}\PY{p}{)}
         \PY{n}{display}\PY{p}{(}\PY{n}{J}\PY{p}{)}
         \PY{n}{display}\PY{p}{(}\PY{n}{P}\PY{p}{)}
         \PY{n}{display}\PY{p}{(}\PY{n}{P} \PY{o}{*} \PY{n}{J} \PY{o}{*} \PY{n}{P}\PY{o}{.}\PY{n}{inv}\PY{p}{(}\PY{p}{)}\PY{p}{)}
\end{Verbatim}


    $$\displaystyle{\begin{pmatrix}a_{1} & 0 & 0 & 0 \\0 & - a_{1} & 0 & 0 \\0 & 0 & -2 & 1 \\0 & 0 & 0 & -2\end{pmatrix}}\; \text{ where } a_{1} = i$$

    
    $$\displaystyle{\begin{pmatrix}\frac{3 a_{1}-11}{5} & \frac{-3 a_{1}-11}{5} & 7 & -\frac{3}{2} \\\frac{11 a_{1} + 68}{65} & \frac{-11 a_{1} + 68}{65} & -2 & 0 \\\frac{-6 a_{1}-17}{13} & \frac{6 a_{1}-17}{13} & 0 & 0 \\1 & 1 & 0 & 1\end{pmatrix}}\; \text{ where } a_{1} = i$$

    
    $$\displaystyle{\begin{pmatrix}20 & 77 & 59 & 40 \\0 & -2 & -3 & -2 \\-10 & -35 & -23 & -15 \\2 & 7 & 3 & 1\end{pmatrix}}$$

    
    We construct a simple matrix and compute its matrix logarithm, which is
uses Jordan decomposition internally:

    \begin{Verbatim}[commandchars=\\\{\}]
{\color{incolor}In [{\color{incolor}76}]:} \PY{n}{A} \PY{o}{=} \PY{n}{ca\PYZus{}mat}\PY{p}{(}\PY{p}{[}\PY{p}{[}\PY{o}{\PYZhy{}}\PY{l+m+mi}{1}\PY{p}{,} \PY{o}{\PYZhy{}}\PY{l+m+mi}{2}\PY{p}{]}\PY{p}{,} \PY{p}{[}\PY{l+m+mi}{1}\PY{p}{,} \PY{l+m+mi}{1}\PY{p}{]}\PY{p}{]}\PY{p}{)}
         \PY{n}{display}\PY{p}{(}\PY{n}{A}\PY{p}{)}
         \PY{n}{display}\PY{p}{(}\PY{n}{A}\PY{o}{.}\PY{n}{log}\PY{p}{(}\PY{p}{)}\PY{p}{)}
\end{Verbatim}


    $$\displaystyle{\begin{pmatrix}-1 & -2 \\1 & 1\end{pmatrix}}$$

    
    $$\displaystyle{\begin{pmatrix}-\frac{ a_{1}}{2} & - a_{1} \\\frac{a_{1}}{2} & \frac{a_{1}}{2}\end{pmatrix}}\; \text{ where } a_{1} = \pi,\;a_{2} = i$$

    
    We evaluate the exponential of the logarithm, recovering the original
matrix. This will only work in very simple cases.

    \begin{Verbatim}[commandchars=\\\{\}]
{\color{incolor}In [{\color{incolor}77}]:} \PY{n}{A}\PY{o}{.}\PY{n}{log}\PY{p}{(}\PY{p}{)}\PY{o}{.}\PY{n}{exp}\PY{p}{(}\PY{p}{)}
\end{Verbatim}

\texttt{\color{outcolor}Out[{\color{outcolor}77}]:}
    
    $$\displaystyle{\begin{pmatrix}-1 & -2 \\1 & 1\end{pmatrix}}$$

    

    Another nice example:

    \begin{Verbatim}[commandchars=\\\{\}]
{\color{incolor}In [{\color{incolor}78}]:} \PY{n}{B} \PY{o}{=} \PY{n}{ca\PYZus{}mat}\PY{p}{(}\PY{p}{[}\PY{p}{[}\PY{l+m+mi}{0}\PY{p}{,}\PY{l+m+mi}{0}\PY{p}{,}\PY{l+m+mi}{1}\PY{p}{]}\PY{p}{,}\PY{p}{[}\PY{l+m+mi}{0}\PY{p}{,}\PY{l+m+mi}{1}\PY{p}{,}\PY{l+m+mi}{0}\PY{p}{]}\PY{p}{,}\PY{p}{[}\PY{l+m+mi}{1}\PY{p}{,}\PY{l+m+mi}{0}\PY{p}{,}\PY{l+m+mi}{0}\PY{p}{]}\PY{p}{]}\PY{p}{)}
         \PY{n}{display}\PY{p}{(}\PY{n}{B}\PY{o}{.}\PY{n}{log}\PY{p}{(}\PY{p}{)}\PY{p}{)}
         \PY{n}{display}\PY{p}{(}\PY{n}{B}\PY{o}{.}\PY{n}{log}\PY{p}{(}\PY{p}{)}\PY{o}{.}\PY{n}{exp}\PY{p}{(}\PY{p}{)}\PY{p}{)}
\end{Verbatim}


    $$\displaystyle{\begin{pmatrix}\frac{a_{1} a_{2}}{2} & 0 & -\frac{ a_{1} a_{2}}{2} \\0 & 0 & 0 \\-\frac{ a_{1} a_{2}}{2} & 0 & \frac{a_{1} a_{2}}{2}\end{pmatrix}}\; \text{ where } a_{1} = \pi,\;a_{2} = i$$

    
    $$\displaystyle{\begin{pmatrix}0 & 0 & 1 \\0 & 1 & 0 \\1 & 0 & 0\end{pmatrix}}$$

    
    The following matrix is nilpotent; its exponential is a polynomial
expression of the matrix:

    \begin{Verbatim}[commandchars=\\\{\}]
{\color{incolor}In [{\color{incolor}79}]:} \PY{n}{A} \PY{o}{=} \PY{n}{ca\PYZus{}mat}\PY{p}{(}\PY{p}{[}\PY{p}{[}\PY{l+m+mi}{10}\PY{p}{,}\PY{l+m+mi}{32}\PY{p}{,}\PY{l+m+mi}{3}\PY{p}{,}\PY{o}{\PYZhy{}}\PY{l+m+mi}{13}\PY{p}{]}\PY{p}{,} \PY{p}{[}\PY{o}{\PYZhy{}}\PY{l+m+mi}{8}\PY{p}{,}\PY{o}{\PYZhy{}}\PY{l+m+mi}{30}\PY{p}{,}\PY{o}{\PYZhy{}}\PY{l+m+mi}{4}\PY{p}{,}\PY{l+m+mi}{11}\PY{p}{]}\PY{p}{,} \PY{p}{[}\PY{l+m+mi}{25}\PY{p}{,}\PY{l+m+mi}{90}\PY{p}{,}\PY{l+m+mi}{11}\PY{p}{,}\PY{o}{\PYZhy{}}\PY{l+m+mi}{34}\PY{p}{]}\PY{p}{,} \PY{p}{[}\PY{o}{\PYZhy{}}\PY{l+m+mi}{6}\PY{p}{,}\PY{o}{\PYZhy{}}\PY{l+m+mi}{28}\PY{p}{,}\PY{o}{\PYZhy{}}\PY{l+m+mi}{5}\PY{p}{,}\PY{l+m+mi}{9}\PY{p}{]}\PY{p}{]}\PY{p}{)}
         \PY{n}{display}\PY{p}{(}\PY{n}{A}\PY{o}{.}\PY{n}{exp}\PY{p}{(}\PY{p}{)}\PY{p}{)}
         \PY{n}{display}\PY{p}{(}\PY{n}{A}\PY{o}{*}\PY{o}{*}\PY{l+m+mi}{0} \PY{o}{+} \PY{n}{A}\PY{o}{*}\PY{o}{*}\PY{l+m+mi}{1} \PY{o}{+} \PY{n}{A}\PY{o}{*}\PY{o}{*}\PY{l+m+mi}{2}\PY{o}{/}\PY{l+m+mi}{2} \PY{o}{+} \PY{n}{A}\PY{o}{*}\PY{o}{*}\PY{l+m+mi}{3}\PY{o}{/}\PY{l+m+mi}{6}\PY{p}{)}
\end{Verbatim}


    $$\displaystyle{\begin{pmatrix}\frac{19}{2} & 29 & 3 & -\frac{23}{2} \\-\frac{21}{2} & -40 & -\frac{11}{2} & 15 \\\frac{57}{2} & 109 & 15 & -\frac{81}{2} \\-\frac{25}{2} & -53 & -8 & \frac{39}{2}\end{pmatrix}}$$

    
    $$\displaystyle{\begin{pmatrix}\frac{19}{2} & 29 & 3 & -\frac{23}{2} \\-\frac{21}{2} & -40 & -\frac{11}{2} & 15 \\\frac{57}{2} & 109 & 15 & -\frac{81}{2} \\-\frac{25}{2} & -53 & -8 & \frac{39}{2}\end{pmatrix}}$$

    
    We construct the 5x5 Hilbert matrix and check that its eigenvalues
satisfy the expected determinant and trace relations:

    \begin{Verbatim}[commandchars=\\\{\}]
{\color{incolor}In [{\color{incolor}80}]:} \PY{n}{H} \PY{o}{=} \PY{n}{ca\PYZus{}mat}\PY{p}{(}\PY{p}{[}\PY{p}{[}\PY{n}{ca}\PY{p}{(}\PY{l+m+mi}{1}\PY{p}{)}\PY{o}{/}\PY{p}{(}\PY{n}{i}\PY{o}{+}\PY{n}{j}\PY{o}{+}\PY{l+m+mi}{1}\PY{p}{)} \PY{k}{for} \PY{n}{i} \PY{o+ow}{in} \PY{n+nb}{range}\PY{p}{(}\PY{l+m+mi}{5}\PY{p}{)}\PY{p}{]} \PY{k}{for} \PY{n}{j} \PY{o+ow}{in} \PY{n+nb}{range}\PY{p}{(}\PY{l+m+mi}{5}\PY{p}{)}\PY{p}{]}\PY{p}{)}
         \PY{n}{H}
\end{Verbatim}

\texttt{\color{outcolor}Out[{\color{outcolor}80}]:}
    
    $$\displaystyle{\begin{pmatrix}1 & \frac{1}{2} & \frac{1}{3} & \frac{1}{4} & \frac{1}{5} \\\frac{1}{2} & \frac{1}{3} & \frac{1}{4} & \frac{1}{5} & \frac{1}{6} \\\frac{1}{3} & \frac{1}{4} & \frac{1}{5} & \frac{1}{6} & \frac{1}{7} \\\frac{1}{4} & \frac{1}{5} & \frac{1}{6} & \frac{1}{7} & \frac{1}{8} \\\frac{1}{5} & \frac{1}{6} & \frac{1}{7} & \frac{1}{8} & \frac{1}{9}\end{pmatrix}}$$

    

    \begin{Verbatim}[commandchars=\\\{\}]
{\color{incolor}In [{\color{incolor}81}]:} \PY{n}{eig} \PY{o}{=} \PY{n}{H}\PY{o}{.}\PY{n}{eigenvalues}\PY{p}{(}\PY{p}{)}
         \PY{k}{for} \PY{n}{c}\PY{p}{,} \PY{n}{mult} \PY{o+ow}{in} \PY{n}{eig}\PY{p}{:}
             \PY{n}{display}\PY{p}{(}\PY{n}{c}\PY{p}{)}
\end{Verbatim}


    $$a_{1}\; \text{ where } a_{1} = \left(\text{Root }\, x \approx {1.56705} \;\text{ of } \;{266716800000 x^{5}-476703360000 x^{4}+92708406000 x^{3}-1022881200 x^{2}+307505 x-1}\right)$$

    
    $$a_{1}\; \text{ where } a_{1} = \left(\text{Root }\, x \approx {0.208534} \;\text{ of } \;{266716800000 x^{5}-476703360000 x^{4}+92708406000 x^{3}-1022881200 x^{2}+307505 x-1}\right)$$

    
    $$a_{1}\; \text{ where } a_{1} = \left(\text{Root }\, x \approx {0.0114075} \;\text{ of } \;{266716800000 x^{5}-476703360000 x^{4}+92708406000 x^{3}-1022881200 x^{2}+307505 x-1}\right)$$

    
    $$a_{1}\; \text{ where } a_{1} = \left(\text{Root }\, x \approx {0.000305898} \;\text{ of } \;{266716800000 x^{5}-476703360000 x^{4}+92708406000 x^{3}-1022881200 x^{2}+307505 x-1}\right)$$

    
    $$a_{1}\; \text{ where } a_{1} = \left(\text{Root }\, x \approx {3.28793 \cdot 10^{-6}} \;\text{ of } \;{266716800000 x^{5}-476703360000 x^{4}+92708406000 x^{3}-1022881200 x^{2}+307505 x-1}\right)$$

    
    \begin{Verbatim}[commandchars=\\\{\}]
{\color{incolor}In [{\color{incolor}82}]:} \PY{n}{display}\PY{p}{(}\PY{n+nb}{sum}\PY{p}{(}\PY{n}{c} \PY{o}{*} \PY{n}{mult} \PY{k}{for} \PY{p}{(}\PY{n}{c}\PY{p}{,} \PY{n}{mult}\PY{p}{)} \PY{o+ow}{in} \PY{n}{eig}\PY{p}{)}\PY{p}{)}\PY{p}{;} \PY{n}{display}\PY{p}{(}\PY{n}{H}\PY{o}{.}\PY{n}{trace}\PY{p}{(}\PY{p}{)}\PY{p}{)}
         \PY{n}{display}\PY{p}{(}\PY{n}{prod}\PY{p}{(}\PY{n}{c} \PY{o}{*}\PY{o}{*} \PY{n}{mult} \PY{k}{for} \PY{p}{(}\PY{n}{c}\PY{p}{,} \PY{n}{mult}\PY{p}{)} \PY{o+ow}{in} \PY{n}{eig}\PY{p}{)}\PY{p}{)}\PY{p}{;} \PY{n}{display}\PY{p}{(}\PY{n}{H}\PY{o}{.}\PY{n}{det}\PY{p}{(}\PY{p}{)}\PY{p}{)}
\end{Verbatim}


    $$\frac{563}{315}$$

    
    $$\frac{563}{315}$$

    
    $$\frac{1}{266716800000}$$

    
    $$\frac{1}{266716800000}$$

    
    \subsection{Polynomials}\label{polynomials}

The \texttt{ca\_poly} type represents univariate polynomials with
\texttt{ca} coefficients. We can construct polynomials and do
arithmetic:

    \begin{Verbatim}[commandchars=\\\{\}]
{\color{incolor}In [{\color{incolor}83}]:} \PY{n}{x} \PY{o}{=} \PY{n}{ca\PYZus{}poly}\PY{p}{(}\PY{p}{[}\PY{l+m+mi}{0}\PY{p}{,}\PY{l+m+mi}{1}\PY{p}{]}\PY{p}{)}
         \PY{n}{f} \PY{o}{=} \PY{l+m+mi}{1} \PY{o}{+} \PY{p}{(}\PY{l+m+mi}{2} \PY{o}{+} \PY{n}{ca}\PY{p}{(}\PY{l+m+mi}{2}\PY{p}{)}\PY{o}{.}\PY{n}{sqrt}\PY{p}{(}\PY{p}{)} \PY{o}{*} \PY{n}{ca}\PY{o}{.}\PY{n}{pi}\PY{p}{(}\PY{p}{)}\PY{p}{)} \PY{o}{*} \PY{n}{x} \PY{o}{+} \PY{l+m+mi}{3}\PY{o}{*}\PY{n}{x}\PY{o}{*}\PY{o}{*}\PY{l+m+mi}{2}
         \PY{n}{f}
\end{Verbatim}

\texttt{\color{outcolor}Out[{\color{outcolor}83}]:}
    
    $$3 x^{2}+ \left(a_{1} a_{2} + 2\right) x+1\; \text{ where } a_{1} = \pi,\;a_{2} = \sqrt{2}$$

    

    \begin{Verbatim}[commandchars=\\\{\}]
{\color{incolor}In [{\color{incolor}84}]:} \PY{n}{f} \PY{o}{*}\PY{o}{*} \PY{l+m+mi}{5}
\end{Verbatim}

\texttt{\color{outcolor}Out[{\color{outcolor}84}]:}
    
    $$243 x^{10}+ \left(405 a_{1} a_{2} + 810\right) x^{9}+ \left(540 a^{2}_{1} + 1080 a_{1} a_{2} + 1485\right) x^{8}+ \left(180 a^{3}_{1} a_{2} + 1080 a^{2}_{1} + 1620 a_{1} a_{2} + 1800\right) x^{7}+ \left(60 a^{4}_{1} + 240 a^{3}_{1} a_{2} + 1260 a^{2}_{1} + 1560 a_{1} a_{2} + 1590\right) x^{6}+ \left(4 a^{5}_{1} a_{2} + 40 a^{4}_{1} + 200 a^{3}_{1} a_{2} + 880 a^{2}_{1} + 1070 a_{1} a_{2} + 1052\right) x^{5}+ \left(20 a^{4}_{1} + 80 a^{3}_{1} a_{2} + 420 a^{2}_{1} + 520 a_{1} a_{2} + 530\right) x^{4}+ \left(20 a^{3}_{1} a_{2} + 120 a^{2}_{1} + 180 a_{1} a_{2} + 200\right) x^{3}+ \left(20 a^{2}_{1} + 40 a_{1} a_{2} + 55\right) x^{2}+ \left(5 a_{1} a_{2} + 10\right) x+1\; \text{ where } a_{1} = \pi,\;a_{2} = \sqrt{2}$$

    

    \begin{Verbatim}[commandchars=\\\{\}]
{\color{incolor}In [{\color{incolor}85}]:} \PY{n}{ca\PYZus{}poly}\PY{p}{(}\PY{p}{[}\PY{l+m+mi}{1}\PY{p}{,}\PY{l+m+mi}{1}\PY{p}{,}\PY{l+m+mi}{1}\PY{p}{,}\PY{l+m+mi}{1}\PY{p}{,}\PY{l+m+mi}{1}\PY{p}{,}\PY{l+m+mi}{1}\PY{p}{,}\PY{l+m+mi}{1}\PY{p}{,}\PY{l+m+mi}{1}\PY{p}{,}\PY{l+m+mi}{1}\PY{p}{]}\PY{p}{)}\PY{o}{.}\PY{n}{integral}\PY{p}{(}\PY{p}{)}
\end{Verbatim}

\texttt{\color{outcolor}Out[{\color{outcolor}85}]:}
    
    $$\frac{1}{9} x^{9} + \frac{1}{8} x^{8} + \frac{1}{7} x^{7} + \frac{1}{6} x^{6} + \frac{1}{5} x^{5} + \frac{1}{4} x^{4} + \frac{1}{3} x^{3} + \frac{1}{2} x^{2}+ x$$

    

    Polynomial division, GCD and other operations work as expected:

    \begin{Verbatim}[commandchars=\\\{\}]
{\color{incolor}In [{\color{incolor}86}]:} \PY{n}{display}\PY{p}{(}\PY{n}{f} \PY{o}{*}\PY{o}{*} \PY{l+m+mi}{5} \PY{o}{/}\PY{o}{/} \PY{n}{f}\PY{o}{*}\PY{o}{*}\PY{l+m+mi}{4}\PY{p}{)}
\end{Verbatim}


    $$3 x^{2}+ \left(a_{1} a_{2} + 2\right) x+1\; \text{ where } a_{1} = \pi,\;a_{2} = \sqrt{2}$$

    
    \begin{Verbatim}[commandchars=\\\{\}]
{\color{incolor}In [{\color{incolor}87}]:} \PY{n}{display}\PY{p}{(}\PY{n}{f}\PY{o}{*}\PY{o}{*}\PY{l+m+mi}{5} \PY{o}{\PYZpc{}} \PY{n}{f}\PY{o}{*}\PY{o}{*}\PY{l+m+mi}{4}\PY{p}{)}
\end{Verbatim}


    $$0$$

    
    \begin{Verbatim}[commandchars=\\\{\}]
{\color{incolor}In [{\color{incolor}88}]:} \PY{p}{(}\PY{n}{f} \PY{o}{*} \PY{p}{(}\PY{n}{x}\PY{o}{*}\PY{o}{*}\PY{l+m+mi}{2}\PY{o}{\PYZhy{}}\PY{l+m+mi}{1}\PY{p}{)}\PY{p}{)}\PY{o}{.}\PY{n}{gcd}\PY{p}{(}\PY{n}{f}\PY{o}{*}\PY{o}{*}\PY{l+m+mi}{2} \PY{o}{*} \PY{p}{(}\PY{n}{x}\PY{o}{\PYZhy{}}\PY{l+m+mi}{1}\PY{p}{)}\PY{p}{)}
\end{Verbatim}

\texttt{\color{outcolor}Out[{\color{outcolor}88}]:}
    
    $$ x^{3} + \frac{2 a^{2}_{1}-3 a_{1} a_{2} + 2}{3 a_{1} a_{2}-6} x^{2} + \frac{-2 a^{2}_{1} + a_{1} a_{2} + 2}{3 a_{1} a_{2}-6} x-\frac{1}{3}\; \text{ where } a_{1} = \pi,\;a_{2} = \sqrt{2}$$

    

    \begin{Verbatim}[commandchars=\\\{\}]
{\color{incolor}In [{\color{incolor}89}]:} \PY{p}{(}\PY{n}{x}\PY{o}{*}\PY{o}{*}\PY{l+m+mi}{2} \PY{o}{+} \PY{n}{ca}\PY{o}{.}\PY{n}{pi}\PY{p}{(}\PY{p}{)}\PY{o}{*}\PY{o}{*}\PY{l+m+mi}{2}\PY{p}{)}\PY{o}{.}\PY{n}{gcd}\PY{p}{(}\PY{n}{x} \PY{o}{+} \PY{n}{ca}\PY{o}{.}\PY{n}{i}\PY{p}{(}\PY{p}{)} \PY{o}{*} \PY{n}{ca}\PY{o}{.}\PY{n}{pi}\PY{p}{(}\PY{p}{)}\PY{p}{)}
\end{Verbatim}

\texttt{\color{outcolor}Out[{\color{outcolor}89}]:}
    
    $$ x + a_{1} a_{2}\; \text{ where } a_{1} = \pi,\;a_{2} = i$$

    

    \begin{Verbatim}[commandchars=\\\{\}]
{\color{incolor}In [{\color{incolor}90}]:} \PY{n}{ca\PYZus{}poly}\PY{p}{(}\PY{p}{[}\PY{l+m+mi}{9}\PY{p}{,}\PY{l+m+mi}{6}\PY{p}{,}\PY{l+m+mi}{7}\PY{p}{,}\PY{o}{\PYZhy{}}\PY{l+m+mi}{28}\PY{p}{,}\PY{l+m+mi}{12}\PY{p}{]}\PY{p}{)}\PY{o}{.}\PY{n}{squarefree\PYZus{}part}\PY{p}{(}\PY{p}{)}
\end{Verbatim}

\texttt{\color{outcolor}Out[{\color{outcolor}90}]:}
    
    $$ x^{3}-\frac{5}{6} x^{2}-\frac{2}{3} x-\frac{1}{2}$$

    

    \begin{Verbatim}[commandchars=\\\{\}]
{\color{incolor}In [{\color{incolor}91}]:} \PY{c+c1}{\PYZsh{} Squarefree factorization}
         \PY{k}{for} \PY{p}{(}\PY{n}{fac}\PY{p}{,} \PY{n}{mult}\PY{p}{)} \PY{o+ow}{in} \PY{n}{ca\PYZus{}poly}\PY{p}{(}\PY{p}{[}\PY{l+m+mi}{9}\PY{p}{,}\PY{l+m+mi}{6}\PY{p}{,}\PY{l+m+mi}{7}\PY{p}{,}\PY{o}{\PYZhy{}}\PY{l+m+mi}{28}\PY{p}{,}\PY{l+m+mi}{12}\PY{p}{]}\PY{p}{)}\PY{o}{.}\PY{n}{factor\PYZus{}squarefree}\PY{p}{(}\PY{p}{)}\PY{p}{[}\PY{l+m+mi}{1}\PY{p}{]}\PY{p}{:}
             \PY{n+nb}{print}\PY{p}{(}\PY{l+s+sa}{f}\PY{l+s+s2}{\PYZdq{}}\PY{l+s+s2}{Multiplicity }\PY{l+s+si}{\PYZob{}}\PY{n}{mult}\PY{l+s+si}{\PYZcb{}}\PY{l+s+s2}{:}\PY{l+s+s2}{\PYZdq{}}\PY{p}{)}
             \PY{n}{display}\PY{p}{(}\PY{n}{fac}\PY{p}{)}
\end{Verbatim}


    \begin{Verbatim}[commandchars=\\\{\}]
Multiplicity 1:

    \end{Verbatim}

    $$ x^{2} + \frac{2}{3} x + \frac{1}{3}$$

    
    \begin{Verbatim}[commandchars=\\\{\}]
Multiplicity 2:

    \end{Verbatim}

    $$ x-\frac{3}{2}$$

    
    Finding the roots of a high-degree polynomial:

    \begin{Verbatim}[commandchars=\\\{\}]
{\color{incolor}In [{\color{incolor}92}]:} \PY{n}{f} \PY{o}{=} \PY{l+m+mi}{4}\PY{o}{*}\PY{n}{x}\PY{o}{*}\PY{o}{*}\PY{l+m+mi}{7} \PY{o}{+} \PY{l+m+mi}{4}\PY{o}{*}\PY{n}{x}\PY{o}{*}\PY{o}{*}\PY{l+m+mi}{6} \PY{o}{\PYZhy{}} \PY{l+m+mi}{11}\PY{o}{*}\PY{n}{x}\PY{o}{*}\PY{o}{*}\PY{l+m+mi}{5} \PY{o}{\PYZhy{}} \PY{l+m+mi}{16}\PY{o}{*}\PY{n}{x}\PY{o}{*}\PY{o}{*}\PY{l+m+mi}{4} \PY{o}{+} \PY{n}{x}\PY{o}{*}\PY{o}{*}\PY{l+m+mi}{3} \PY{o}{+} \PY{l+m+mi}{15}\PY{o}{*}\PY{n}{x}\PY{o}{*}\PY{o}{*}\PY{l+m+mi}{2} \PY{o}{+} \PY{l+m+mi}{10}\PY{o}{*}\PY{n}{x} \PY{o}{+} \PY{l+m+mi}{2}
         
         \PY{k}{for} \PY{n}{root}\PY{p}{,} \PY{n}{mult} \PY{o+ow}{in} \PY{n}{f}\PY{o}{.}\PY{n}{roots}\PY{p}{(}\PY{p}{)}\PY{p}{:}
             \PY{n+nb}{print}\PY{p}{(}\PY{l+s+sa}{f}\PY{l+s+s2}{\PYZdq{}}\PY{l+s+s2}{Multiplicity }\PY{l+s+si}{\PYZob{}}\PY{n}{mult}\PY{l+s+si}{\PYZcb{}}\PY{l+s+s2}{:}\PY{l+s+s2}{\PYZdq{}}\PY{p}{)}
             \PY{n}{display}\PY{p}{(}\PY{n}{root}\PY{p}{)}
\end{Verbatim}


    \begin{Verbatim}[commandchars=\\\{\}]
Multiplicity 1:

    \end{Verbatim}

    $$a_{1}\; \text{ where } a_{1} = \sqrt{2}$$

    
    \begin{Verbatim}[commandchars=\\\{\}]
Multiplicity 1:

    \end{Verbatim}

    $$a_{1}\; \text{ where } a_{1} = \left(\text{Root }\, x \approx {1.32472} \;\text{ of } \;{ x^{3}- x-1}\right)$$

    
    \begin{Verbatim}[commandchars=\\\{\}]
Multiplicity 1:

    \end{Verbatim}

    $$- a_{1}\; \text{ where } a_{1} = \sqrt{2}$$

    
    \begin{Verbatim}[commandchars=\\\{\}]
Multiplicity 1:

    \end{Verbatim}

    $$a_{1}\; \text{ where } a_{1} = \left(\text{Root }\, x \approx {-0.662359 + 0.562280 i} \;\text{ of } \;{ x^{3}- x-1}\right)$$

    
    \begin{Verbatim}[commandchars=\\\{\}]
Multiplicity 1:

    \end{Verbatim}

    $$a_{1}\; \text{ where } a_{1} = \left(\text{Root }\, x \approx {-0.662359-0.562280 i} \;\text{ of } \;{ x^{3}- x-1}\right)$$

    
    \begin{Verbatim}[commandchars=\\\{\}]
Multiplicity 2:

    \end{Verbatim}

    $$-\frac{1}{2}$$

    
    \begin{Verbatim}[commandchars=\\\{\}]
{\color{incolor}In [{\color{incolor}93}]:} \PY{n}{f}\PY{p}{(}\PY{n}{sqrt}\PY{p}{(}\PY{l+m+mi}{2}\PY{p}{)}\PY{p}{)}
\end{Verbatim}

\texttt{\color{outcolor}Out[{\color{outcolor}93}]:}
    
    $$0$$

    


    % Add a bibliography block to the postdoc
    
    
    
    \end{document}
